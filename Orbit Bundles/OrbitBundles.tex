\documentclass[10pt, oneside]{article} 
\usepackage{mathtools,mathrsfs, amsmath, amsthm, amssymb, wasysym, verbatim, bbm, color, graphics, geometry, xargs, hyperref}
\usepackage[pdftex,dvipsnames]{xcolor}

\usepackage[colorinlistoftodos,prependcaption,textsize=tiny]{todonotes}

\newcommandx{\fix}[2][1=]{\todo[linecolor=red,backgroundcolor=red!25,bordercolor=red,#1]{#2}}
\newcommandx{\improve}[2][1=]{\todo[linecolor=blue,backgroundcolor=blue!25,bordercolor=blue,#1]{#2}}
\newcommandx{\note}[2][1=]{\todo[linecolor=OliveGreen,backgroundcolor=OliveGreen!25,bordercolor=OliveGreen,#1]{#2}}
\newcommandx{\unsure}[2][1=]{\todo[linecolor=Plum,backgroundcolor=Plum!25,bordercolor=Plum,#1]{#2}}

\geometry{tmargin=.75in, bmargin=1in, lmargin=1in, rmargin = 1in}  

%% NOTATION
%%%%%
\newcommand{\R}{\mathbb{R}}
\newcommand{\C}{\mathbb{C}}
\newcommand{\Z}{\mathbb{Z}}
\newcommand{\N}{\mathbb{N}}
\newcommand{\Q}{\mathbb{Q}}
\newcommand{\Cdot}{\boldsymbol{\cdot}}
\newcommand{\SO}{\text{SO}(2)}
\newcommand{\homeo}[1][S^1]{\text{Homeo}_0(#1)}
\newcommand{\cl}[1]{\overline{#1}}
\newcommand{\conf}[2][S^1]{\text{Conf}_{#2}(#1)}
\newcommand{\pconf}[2][S^1]{\text{PConf}_{#2}(#1)}

%% CUSTOM THEOREMS
%%%%%
\newtheorem{thm}{Theorem}[section]
\newtheorem*{thm*}{Theorem}
\theoremstyle{definition}
\newtheorem{defn}{Definition}[section]
\newtheorem{conv}{Convention}[section]
\newtheorem{conj}{Conjecture}[section]
\newtheorem{rem}{Remark}[section]
\newtheorem*{obs*}{Observation}
\newtheorem{lem}{Lemma}[section]
\newtheorem{cor}{Corollary}[section]
\newtheorem*{fact*}{Fact}
\newtheorem{prop}{Proposition}[section]
\newtheorem*{clm*}{Claim}
\theoremstyle{definition}
\newtheorem*{prog*}{Program}

\title{Orbit Bundle Theorem and Applications}
\author{Hazel Brenner}
\date{Spring 2023}
\begin{document}

\maketitle

\section{Introduction}
This writeup is an extension 
of what I've written in No Badly Compactified Annuli.
This will prove a general version of the {\it Orbit Bundle Structure} 
which concerns the structure of connected regions consisting of orbits of a fixed dimension. 
As of writing, 
this result takes as an input that the region is a manifold 
(would be ideal to have settings that this can be relaxed). 

The setting of the generalized bundle structure result is the following.  
Suppose $\homeo[M]$ acts on $N$,
and $X$ is a connected submanifold of $N$ 
with orbits of fixed dimension. 
Fix a base point $b\in \conf[M]{n}$. 
We will abuse notation 
by denoting both the point in $\conf[M]{n}$ 
and the corresponding subset of $M$ as $b$.
Also, 
note that the deck group of the maximal admissible covering $C_b \to \conf[M]{n}$ can be identified with $\text{stab}(b)/\text{stab}_0(b)\subset\homeo[M]$.

\begin{prop}[Orbit Bundle]
    Suppose $A$ a connected submanifold of $N$ 
    with orbits of fixed dimension, 
    then there is a continuous map $p: A\to \conf[M]{n}$ 
    such that there's a map: $$I: C_n \times p^{-1}(b) / \Gamma \to A.$$
    where $\Gamma$ is the deck group of the covering $C_n \to \conf[M]{n}$ 
    acting on the first factor by deck transformations 
    and the second factor by the $\homeo[M]$ action 
    restricted to the subgroup $\Gamma$ acting on the fiber.
    The map $I$ is an equivariant homeomorphism 
    with the $\homeo[M]$ action on $C_n \times p^{-1}(b) / \Gamma$ being the quotient of the product of the standard action on $C_n$ 
    with the trivial action on the fiber.
\end{prop}
\begin{proof}
    First, 
    suppose $O$ is an orbit in $A$. 
    Recall that $O$ is a continuous injective image 
    of an admissible cover of $\conf[M]{n}$ 
    with a lift of the standard configuration action. 
    So, in particular, 
    for every point $x\in O$, 
    there is a unique $X\in\conf[M]{n}$ 
    such that $\text{stab}_0(X) \subseteq G_x \subseteq \text{stab}(X)$. 
    Then, define $p(x) = X$. 
    Since $G_{\rho(f)(x)} = f G_x f^{-1}$ 
    it follows that $\text{stab}_0(f(X)) \subseteq G_{\rho(f)(x)}\subseteq \text{stab}(f(X))$. 
    This directly implies $p$ is equivariant. 

    Next we wish to show that $p$ is continuous.
    We will do this by checking continuity on a nice basis. 
    In particular, 
    note that if $\mathscr{B}$ is a basis for the topology on $M$, 
    then the set of all products of $n$ disjoint elements of $\mathscr{B}$ form a basis for the topology on $\pconf[M]{n}$ 
    i.e., the $n$-fold product of $M$ with itself minus the fat diagonal. 
    Then, the image of this basis under the quotient forms a basis for the topology on $\conf[M]{n}$.
    Let $U$ be image of one such $B_1\times\dots\times B_n$ 
    under the quotient map to $\conf[M]{n}$. 

    \begin{clm*}
        Let $H(B_i)$ denote the $\homeo[M]$ subgroup of homeomorphisms supported on $B_i$.
        $$p^{-1}(U) = A \setminus \bigcup \text{fix}(H(B_i))$$
    \end{clm*}
    \begin{proof}
        By the equivariance of $p$ 
        and the \textit{locally continuously transitive} property of homeomorphism group actions
        $$\text{fix}(H(B_i)) = \{x\in A\, |\, p(x)\cap B_i = \emptyset\}$$
        where $p(x)$ is being interpreted as a subset of $M$. Interpreted semantically, 
        this says that 
        the fixed point set of $H(B_i)$ acting on $A$ is 
        the set of all points whose image under $p$ has no coordinate in $B_i$. 
        Then, by making purely set-theoretic manipulations, 
        we have the following chain of equivalences:
        \begin{align*}
            A - \cup\text{fix}(H(B_i))
        \end{align*}
    \end{proof}
    
\end{proof}

\section{Codimension 1 Orbit Bundles}
For components of codimension 1 orbits, 
the fiber must be 1-dimensional 
and all 1-dimensional homology manifolds are manifolds. 
So, 
codimension 1 orbit bundlses are classified 
by actions of the deck group of $\Gamma$ 
of the covering $C_X \to \conf[M]{n}$ on 1-manifolds 
up to equivariant homeomorphism.
Note that, 
in general, $\lvert \pi_0(F) \rvert \leq \lvert \Gamma \rvert$ 
and the components of $F$ are homeomorphic. 

It's hard to say something overly general here, 
but fixing $M = S^1$ 
and $\text{dim}(N)=3$ a classification is possible. 
Note, in this setting $C_X = \pconf{2}$ and $\Gamma = \Z_2$.

\begin{prop}
    The classification of $\homeo$ codimension 1 orbit bundles $X$ in 3-manifolds is as follows. 
    \begin{description}
        \item[$F$ disconnected] $F\cong C\sqcup C$ for a 1-manifold $C$, $X\cong \pconf{2}\times C$ and $\pi_{SO(2)}(X) \cong I \times C$
        \item[$F$ connected] \ 
        \begin{description}
            \item[$\Z_2$ acts on $F$ f.p. free] $F\cong S^1$, action is by a $\pi$ rotation. $X\cong \pconf{2} \times S^1$ and $\pi_{SO(2)} \cong I\times S^1$
            \item[$\Z_2$ acts on $F$ w/ fixed points] This depends on $F$, on each $\Z_2$ acts by reflection. Note in each case, the bundle is described in terms of the Seifert structure given by the $SO(2)$ action.
            \begin{description}
                \item[$F\cong \R$] $X$ is $(\mathring{\mathbb{D}}^2, (2, 1))$
                \item[$F\cong S^1$] $X$ is $(\mathring{\mathbb{D}}^2, (2, 1), (2,1))$
                \item[$F \cong \lbrack 0,1\rbrack$] $X$ is $(\mathring{\mathbb{D}}^2\cup I, (2,1))$ where $I$ is a interval in the boundary of the closed disk $\mathbb{D}^2$
            \end{description}
            \item[$\Z_2$ acts trivially] $X\cong\conf{2}\times C$ {\it N.B. resulting bundle is nonorientable (and top dimensional), so $N$ is as well} 
        \end{description}
    \end{description}
    A generic component $C$ of the fiber is one of $S^1,\, \mathbb{R},\, [0,1]\, \text{or}\,  [0,1)$
\end{prop}

\begin{proof}
    Note, 
    the quotient gives a 2-fold covering 
    from $\pconf{2}\times F$ 
    to the orbit bundle $X$, 
    which is connected, 
    so $\pconf{2}\times F$ 
    and thus $F$ has at most two components. 
    Assume $F$ has two components. 
    Since the quotient is connected, 
    the action must permute the components (homeomorphically), 
    so in particular we can express the fiber 
    as $F\cong C\sqcup C$. 
    Then the action is just a homeomorphism 
    from one compenent of $\pconf{2}\times F$ to the other, 
    so the quotient is $\pconf{2}\times C$ 
    where the resulting $\homeo$ action is the product of the standard configuration space action on $\pconf{2}$ and the trivial action on the fiber $C$.

    If $F$ has a single component, 
    then the $\Z_2$ action is simply a choice of involution on $F$. For any possible $F$, 
    we can allow $\Z_2$ to act trivially. 
    Then, 
    the quotient loads entirely 
    onto the $\pconf{2}$ factor, 
    where $\Z_2$ acts as the deck group of $\pconf{2}\to\conf{2}$, 
    so the resulting quotient is $\conf{2}\times F$. 
    This is nonorientable (regardless of $F$) and top-dimensional, 
    so in general, 
    such orbit bundles only arise when $N$ is nonorientable. 
    The remainder of the proof is done by exhaustion. 
    $[0,1)$ has no nontrivial involutions, 
    $[0,1]$ and $\mathbb{R}$ each have one 
    and $S^1$ has two. 
    In each case, 
    the quotient preserves the product structure 
    away from fixed points 
    and sends every $\pconf{2} \times \{x\}$ 
    for fixed point $x$ 
    to $\conf{2}\times \{x\}$. 
    The result in each case is trivially the Seifert structure described in the proposition.
\end{proof}

\end{document}