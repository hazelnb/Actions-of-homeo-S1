\documentclass[10pt, oneside]{article} 
\usepackage{mathtools,mathrsfs, amsmath, amsthm, amssymb, wasysym, verbatim, bbm, color, graphics, geometry, xargs, hyperref}
\usepackage[pdftex,dvipsnames]{xcolor}

\usepackage[colorinlistoftodos,prependcaption,textsize=tiny]{todonotes}

\newcommandx{\fix}[2][1=]{\todo[linecolor=red,backgroundcolor=red!25,bordercolor=red,#1]{#2}}
\newcommandx{\improve}[2][1=]{\todo[linecolor=blue,backgroundcolor=blue!25,bordercolor=blue,#1]{#2}}
\newcommandx{\note}[2][1=]{\todo[linecolor=OliveGreen,backgroundcolor=OliveGreen!25,bordercolor=OliveGreen,#1]{#2}}
\newcommandx{\unsure}[2][1=]{\todo[linecolor=Plum,backgroundcolor=Plum!25,bordercolor=Plum,#1]{#2}}

\geometry{tmargin=.75in, bmargin=1in, lmargin=1in, rmargin = 1in}  

%% NOTATION
%%%%%
\newcommand{\R}{\mathbb{R}}
\newcommand{\C}{\mathbb{C}}
\newcommand{\Z}{\mathbb{Z}}
\newcommand{\N}{\mathbb{N}}
\newcommand{\Q}{\mathbb{Q}}
\newcommand{\Cdot}{\boldsymbol{\cdot}}
\newcommand{\SO}{\text{SO}(2)}
\newcommand{\homeo}[1][S^1]{\text{Homeo}_0(#1)}
\newcommand{\cl}[1]{\overline{#1}}
\newcommand{\conf}[2][S^1]{\text{Conf}_{#2}(#1)}
\newcommand{\pconf}[2][S^1]{\text{PConf}_{#2}(#1)}
\newcommand{\set}{{\{\cdot\}}}

%% CUSTOM THEOREMS
%%%%%
\newtheorem{thm}{Theorem}[section]
\newtheorem*{thm*}{Theorem}
\theoremstyle{definition}
\newtheorem{defn}{Definition}[section]
\newtheorem{conv}{Convention}[section]
\newtheorem{conj}{Conjecture}[section]
\newtheorem{ex}{Example}[section]
\newtheorem{rem}{Remark}[section]
\newtheorem*{obs*}{Observation}
\newtheorem{lem}{Lemma}[section]
\newtheorem{cor}{Corollary}[section]
\newtheorem*{fact*}{Fact}
\newtheorem{prop}{Proposition}[section]
\newtheorem*{clm*}{Claim}
\theoremstyle{definition}
\newtheorem*{prog*}{Program}

\title{Orbit Bundle Theorem and Applications}
\author{Hazel Brenner}
\date{Spring 2023}
\begin{document}

\maketitle

\section{Introduction}
The following is a jumble
of tools 
and subsequently interesting examples 
of actions of $\homeo$ 
on three-manifolds


\section{Tools}
\subsection{Gluing}
\begin{prop}
    Suppose $\rho$ and $\rho'$ are actions 
    of $\homeo[M]$ 
    on $N$ and $N'$
    which are manifolds
    of the same dimension
    with boundary,
    and further let $X\subseteq\partial N$ be invariant 
    with $\varphi: X \to \partial N'$ an embedding 
    such that $\rho$ and $\rho'$ are conjugate 
    via $\varphi$. 
    Then, 
    there is an action 
    of $\homeo[M]$ 
    on $N\cup_\varphi N'$ 
    which restricts 
    to $\rho$ and $\rho'$ 
    on $N$ and $N'$ respectively.
\end{prop}
\begin{proof}
    Use the pasting lemma.
\end{proof}


\subsection{Products}
\begin{prop}\label{product-construction}
    Suppose $\rho$ and $\rho'$ are actions 
    of $\homeo[M]$ 
    on $A$ and $B$.
    Then,
    there is an action 
    denoted $\rho\times\rho'$
    of $\homeo[M]$ 
    on $A\times B$
    given by
    $$\rho\times\rho'(f)(x,y) = (\rho(f)(x), \rho(f)(y))$$
\end{prop}


\subsection{Quotients}
\begin{prop}
    Suppose 
    that $q: X\to Y$ is
    a quotient map.
    The set
    $$\Gamma_q \coloneqq\{f\in\homeo[X]\; \vert\; q(x) = q(y) \implies q(f^{\pm 1}(x)) = q(f^{\pm 1}(y))\}$$
    is a subgroup,
    and there is a group homeomorphism
    $q_*: \Gamma_q \to \homeo[Y]$.
\end{prop}

In particular,
\begin{cor}
    If $\rho:\homeo[M]\to\homeo[X]$ is an action,
    such that $\text{im}(\rho)\subseteq \Gamma_q$,
    then there is an action
    of $\homeo[M]$
    on the quotient $Y$
    given by composing $\rho$ and $q_*$.
    We will say
    that such an action
    \textit{descends to the quotient}
\end{cor}
So if $q$ just collapses a set 
of homeomorphic orbits 
to a single orbit
in a ``reasonable'' way,
then the action descends. More precisely,

\begin{cor}
    Suppose $q: X\to Y$ is a quotient map
    such that there is a set $C$
    consisting entirely 
    of homeomorphic orbits 
    such that $q|_{X - C}$ is a homeomorphism onto its image,
    $q(C)$ is a single orbit 
    of the same homeomorphism type
    of the orbits
    in $C$,
    for all orbits $O\in C$, 
    $q|_O$ is a homeomorphism 
    onto its image,
    and for all orbits $O, O'\subseteq C$
    the induced homeomorphism $\varphi_q: O\to O'$ commutes with $p: C \to \conf[M]{n}$, i.e., $p\vert_O = p\vert_{O'}\circ\varphi_q^{-1}$ and  $p\vert_{O'} = p\vert_O\circ\varphi_q$
\end{cor}

\section{Examples}
\subsection{Diagonal action on \texorpdfstring{$T^3$}{T3}}
\subsection{Product Constructions}
\begin{ex}

\end{ex}
\subsection{Quotient Constructions}
\begin{ex}
    Denote by $\Delta:\homeo\to\homeo[\overline{Ann}]$ the action induced by splitting $T^2$ with the configuration space action along the invariant circle. Then, by the product construction in Proposition \ref{product-construction}, there is a continuous action $\Delta \times 0$ of $\homeo$ on $\overline{Ann}\times S^1$ where 0 is the trivial action. Under this action, every circle of the form $(0,\theta) \times S^1$ and $(1, \theta) \times S^1$ is invariant. Then consider the relation $(1, \theta, \varphi) \sim (1, \theta, \varphi)$ for all $\theta\in S^1$. Trivially, for all $f\in\homeo$, then $(\Delta\times 0)(f)(1,\theta, \varphi) = (1, \theta, f(\varphi))$, so $\Delta\times 0$ descends to the quotient. Successively performing the same quotient on $0\times S^1\times S^1$ yields an action on $S^2\times S^1$ with an $Ann\times S^1$ orbit bundles such the positive frontier of all of the annuli orbit are a single invariant $S^1$ and the same with the negative frontiers.
\end{ex}

\end{document}