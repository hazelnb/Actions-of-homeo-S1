\documentclass[10pt, oneside]{article} 
\usepackage{mathtools, amsmath, amsthm, amssymb, calrsfs, wasysym, verbatim, bbm, color, graphics, geometry, xargs}
\usepackage[pdftex,dvipsnames]{xcolor}

\usepackage[colorinlistoftodos,prependcaption,textsize=tiny]{todonotes}

\newcommandx{\fix}[2][1=]{\todo[linecolor=red,backgroundcolor=red!25,bordercolor=red,#1]{#2}}
\newcommandx{\improve}[2][1=]{\todo[linecolor=blue,backgroundcolor=blue!25,bordercolor=blue,#1]{#2}}
\newcommandx{\note}[2][1=]{\todo[linecolor=OliveGreen,backgroundcolor=OliveGreen!25,bordercolor=OliveGreen,#1]{#2}}
\newcommandx{\unsure}[2][1=]{\todo[linecolor=Plum,backgroundcolor=Plum!25,bordercolor=Plum,#1]{#2}}

\geometry{tmargin=.75in, bmargin=1in, lmargin=1in, rmargin = 1in}  

%% NOTATION
%%%%%
\newcommand{\R}{\mathbb{R}}
\newcommand{\C}{\mathbb{C}}
\newcommand{\Z}{\mathbb{Z}}
\newcommand{\N}{\mathbb{N}}
\newcommand{\Q}{\mathbb{Q}}
\newcommand{\Cdot}{\boldsymbol{\cdot}}
\newcommand{\SO}{\text{SO}(2)}
\newcommand{\homeoS}{\text{Homeo}_0(S^1)}
\newcommand{\cl}[1]{\overline{#1}}

%% CUSTOM THEOREMS
%%%%%
\newtheorem{thm}{Theorem}
\newtheorem*{thm*}{Theorem}
\theoremstyle{definition}
\newtheorem{defn}{Definition}
\newtheorem{conv}{Convention}
\newtheorem{conj}{Conjecture}
\newtheorem{rem}{Remark}
\newtheorem*{obs*}{Observation}
\newtheorem{lem}{Lemma}
\newtheorem{cor}{Corollary}
\newtheorem{prop}{Proposition}
\theoremstyle{definition}
\newtheorem*{prog*}{Program}

\usepackage{biblatex}
\addbibresource{InvariantAnnuli.bib}

\title{Invariant Annuli and Singular Fibers}
\author{Hazel Brenner}
\date{Spring 2023}

\begin{document}

\maketitle

\section{Introduction}
The following is a result about the structure of compact three-manifolds admitting fixed point-free $\homeoS$ actions. An important observation is that by considering the restriction of the $\homeoS$ action to rotation subgroup, every fixed point-free $\homeoS$ action on a three-manifold has an associated Seifert fibered structure on that manifold. It is then natural to ask if every Seifert fibered space admits an action of $\homeoS$. We will see by the proposition in the following section that the answer is, in fact, no.

\section{Where Do Singular Fibers Occur}
A first obstruction to the extension of an $\SO$ action to a fixed point-free action of $\homeoS$ on a three manifold is the presence of singular fibers of multiplicity greater than 2 in the associated Seifert fibered structure. The orbits of $\homeoS$ are unions of fibers of the Seifert fibered structure, so any given fiber lies within an orbit or is an orbit itself. We rule out both options for singular as follows.

\begin{obs*}
    Suppose that $\homeoS$ acts without fixed-points on compact three-manifold $M$. Then every fiber of the associated Seifert fibered structure contained in an orbit of $\homeoS$ is regular with the exception of one multiplicity 2 fiber in each orbit which is topologically a M\"{o}bius band.
\end{obs*}

\noindent This follows from \cite{mann-chen}. \note{For the future, it could be useful to write up the classification of $\homeoS$ orbits on three-manifolds using the classification thm in mann-chen.} So in particular, if there are singular fibers of the $\SO$ action, they must be 1-dimensional orbits of the proposed extended action, but we will see from the following proposition that this is not possible.

\begin{prop}
    If $\sigma$ is a 1-dimensional orbit of a fixed point-free action of $\homeoS$ on a compact three-manifold, then $\sigma$ is a regular fiber of the associated Seifert fibration.
\end{prop}

\begin{proof}
   We will consider the restriction of the $\homeoS$ action on $\sigma$ to the subgroup $G_0\subset \homeoS$ where $G_0$ is a point-stabilizer subgroup of $\homeoS$. In particular, the action of $G_0$ on $\sigma$ has some fixed point $x$. If we suppose toward a contradiction that $\sigma$ has multiplicity greater than 1, then there is some angle $\theta$ such that the rotation $r_\theta$ fixes $\sigma$ pointwise. Then, note that $G_\theta = r_\theta G_0 r_\theta^{-1}$ also fixes $x$, but in fact the product of any two point stabilizer subgroups is all of $\homeoS$. There are a number of ways to observe this fact, an easy way is by noting that $\homeoS$ acts doubly transitively on $S^1$, so point stabilizers are maximal subgroups. The consequence of this is that $x$ must be a global fixed point, which is a contradiction. So, in particular, $\sigma$ must be regular.
\end{proof} 

\noindent Taking these two facts together, we see that the Seifert-fibered structure associated to an action of $\homeoS$ on a three-manifold has only singular fibers of multiplicity 2.

\listoftodos[Notes]
\printbibliography
\end{document}