\documentclass[10pt, oneside]{article} 
\usepackage{amsmath, amsthm, amssymb, calrsfs, wasysym, verbatim, bbm, color, graphics, geometry}

\geometry{tmargin=.75in, bmargin=1in, lmargin=1in, rmargin = 1in}  

%% NOTATION
%%%%%
\newcommand{\R}{\mathbb{R}}
\newcommand{\C}{\mathbb{C}}
\newcommand{\Z}{\mathbb{Z}}
\newcommand{\N}{\mathbb{N}}
\newcommand{\Q}{\mathbb{Q}}
\newcommand{\Cdot}{\boldsymbol{\cdot}}
\newcommand{\SO}{\text{SO}(2)}
\newcommand{\homeoS}{\text{Homeo}_0(S^1)}
\newcommand{\cl}[1]{\overline{#1}}

%% CUSTOM THEOREMS
%%%%%
\newtheorem{thm}{Theorem}
\newtheorem*{thm*}{Theorem}
\theoremstyle{definition}
\newtheorem{defn}{Definition}
\newtheorem{conv}{Convention}
\newtheorem{conj}{Conjecture}
\newtheorem{rem}{Remark}
\newtheorem{lem}{Lemma}
\newtheorem{cor}{Corollary}
\newtheorem{prop}{Proposition}
\theoremstyle{definition}
\newtheorem*{prog*}{Program}

% \usepackage{biblatex}
% \addbibresource{main.bib}

\title{Invariant Annuli and Singular Fibers}
\author{Hazel Brenner}
\date{Spring 2023}

\begin{document}

\maketitle

\section{Idea of writeup}

This writeup is the first step toward showing that one dimensional orbits of $\homeoS$ actions on compact three-manifolds are regular fibers of the associated Seifert fibration. In particular, the goal here is to prove that when there is an invariant annulus accumulating onto a given 1-diemnsional orbit, it must be regular. The proof of this fact proceeds by making use of a simple observation about sequences of points in invariant surfaces of $\homeoS$ actions on spaces. 

\section{Main trick}
We have the following fact which more or less amounts to a restatement of continuity.

\begin{rem}
    Suppose that $\homeoS$ acts on a compact three-manifold $M$, and $A$ is an invariant annulus with standard model coordinates $\varphi: S^1\times (0,1) \to A$ and $\sigma$ a 1-dimensional orbit in $\cl{A}- A$. Then, if two sequences of points in $A$ with constant $S^1$ coordinate under $\varphi$ converge in $M$ to $X\in\sigma$, then their $S^1$ coordinate must be the same.
\end{rem}

%\printbibliography
\end{document}