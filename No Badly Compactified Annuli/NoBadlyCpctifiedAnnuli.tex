\documentclass[10pt, oneside]{article} 
\usepackage{mathtools, amsmath, amsthm, amssymb, wasysym, verbatim, bbm, color, graphics, geometry, xargs, hyperref}
\usepackage[pdftex,dvipsnames]{xcolor}

\usepackage[colorinlistoftodos,prependcaption,textsize=tiny]{todonotes}

\newcommandx{\fix}[2][1=]{\todo[linecolor=red,backgroundcolor=red!25,bordercolor=red,#1]{#2}}
\newcommandx{\improve}[2][1=]{\todo[linecolor=blue,backgroundcolor=blue!25,bordercolor=blue,#1]{#2}}
\newcommandx{\note}[2][1=]{\todo[linecolor=OliveGreen,backgroundcolor=OliveGreen!25,bordercolor=OliveGreen,#1]{#2}}
\newcommandx{\unsure}[2][1=]{\todo[linecolor=Plum,backgroundcolor=Plum!25,bordercolor=Plum,#1]{#2}}

\geometry{tmargin=.75in, bmargin=1in, lmargin=1in, rmargin = 1in}  

%% NOTATION
%%%%%
\newcommand{\R}{\mathbb{R}}
\newcommand{\C}{\mathbb{C}}
\newcommand{\Z}{\mathbb{Z}}
\newcommand{\N}{\mathbb{N}}
\newcommand{\Q}{\mathbb{Q}}
\newcommand{\Cdot}{\boldsymbol{\cdot}}
\newcommand{\SO}{\text{SO}(2)}
\newcommand{\homeoS}{\text{Homeo}_0(S^1)}
\newcommand{\cl}[1]{\overline{#1}}
\newcommand{\conf}[1]{\text{Conf}_{#1}(S^1)}
\newcommand{\pconf}[1]{\text{PConf}_{#1}(S^1)}

%% CUSTOM THEOREMS
%%%%%
\newtheorem{thm}{Theorem}[section]
\newtheorem*{thm*}{Theorem}
\theoremstyle{definition}
\newtheorem{defn}{Definition}[section]
\newtheorem{conv}{Convention}[section]
\newtheorem{conj}{Conjecture}[section]
\newtheorem{rem}{Remark}[section]
\newtheorem*{obs*}{Observation}
\newtheorem{lem}{Lemma}[section]
\newtheorem{cor}{Corollary}[section]
\newtheorem*{fact*}{Fact}
\newtheorem{prop}{Proposition}[section]
\theoremstyle{definition}
\newtheorem*{prog*}{Program}

% \usepackage{biblatex}
% \addbibresource{InvariantAnnuli.bib}

\title{Open Regions Foliated by Annulus Orbits}
\author{Hazel Brenner}
\date{Summer-Fall 2023}

\begin{document}

\maketitle

\section{Introduction}
The following is a result about the structure of compact three-manifolds admitting fixed point-free $\homeoS$ actions having only one dimensional and annulus orbits. As a matter of convention, denote we denote configuration space actions and their lifts to covers by $*$. A fact we will make repeated use of to prove these results is the following:

\begin{rem}
    Let $\pi_{SO(2)}$ denote the projection along the orbits of the rotation subgroup to the base orbifold. Then, since $M^3$ and the base orbifold are compact Hausdorff spaces, by a classical fact from point-set topology, $\pi_{SO(2)}$ is a closed map.
\end{rem}

\section{Global structure}
Consider the subset of $M$ consisting of 1-dimensional orbits. The set $\text{fix}(G_0)$ is closed, and $\pi_{SO(2)}$ is a closed map, so $\pi_{SO(2)}(\text{fix}(G_0))$ is closed in the base surface. By continuity then $\pi_{SO(2)}^{-1}(\pi_{SO(2)}(\text{fix}(G_0)))$,which is the union of all of the one-dimensional orbits, is closed.
The complement is then an open subset of $M$ which decomposes into annulus orbits. It is natural to ask about the structure of this open region. Below are two ipmortant structural results:

\section{Foliation by Annuli}

We seek to investigate the structure of an open region consisting of orbits of the same type. For what follows, consider $X$ a connected component of the open region consisiting entirely of annulus orbits. Of primary relevance in our investigation will be the map $p: X\to \text{Conf}_2(S^1)$ given by $p(x) = (\theta, \varphi)$ where $\theta$ and $\varphi$ are the unique points in $S^1$ such that $x\in\text{fix}(G_\theta\cap G_\varphi)$. First, an elementary property of group actions.

\begin{rem}
    Suppose there is an action of group $G$ on $M$ by homeomorphisms and $H$ is a subgroup of $G$. Then, for all $g\in G$,
    $$g * \text{fix}(H) = \text{fix}(g H g^{-1})$$
\end{rem}
\begin{proof}
    Suppose $y\in g * \text{fix}(H)$, then $g^{-1}*y\in \text{fix}(H)$. I.e., $(hg^{-1})*y = g^{-1}*y$ for all $h\in H$. So, by cancellation, $(ghg^{-1})*y = y$ for all $h\in H$; thus, $y\in \text{fix}(g H g^{-1})$. Each step was reversible, so this concludes the proof of equality.
\end{proof}

A similar (but dual) argument shows that for any $f\in\homeoS$ and $G_\theta$ a point stabilizer subgroup, $f G_\theta f^{-1} = G_{f(\theta)}$.

\begin{prop}
    The map $p$ is equivariant, continuous, and its fibers are homeomorphic.
\end{prop}
\begin{proof}
    First, we check equivariance. Suppose $x\in X$, $f\in\homeoS$ and $p(x)=(\theta, \varphi)$. Under the standard configuration space action $f*p(x) = (f(\theta), f(\varphi))$. So we, need to check that $\rho(f)(x)\in \text{fix}(\rho(G_{f(\theta)}\cap G_{f(\varphi)}))$. By the remark, $$\rho(f)(\text{fix}(G_\theta \cap G_\varphi)) = \text{fix}(\rho(f(G_\theta\cap G_\varphi)f^{-1}))$$
    Then, pulling the conjugation through the intersection and applying the second remark,
    $$\rho(f)(\text{fix}(G_\theta \cap G_\varphi)) = \text{fix}(\rho(G_{f(\theta)}\cap G_{f(\varphi)}))$$
    Thus, $f*p(x) = p(\rho(f)(x))$.

    Second, we check continuity by checking that open neighborhoods of points in $\text{conf}_2(S^1)$ pull back to open subsets of $X$. Let $(a, b)$ be a point $\text{conf}_2(S^1)$ and $U = (I, J)$ be a small open neighborhood of $(a,b)$ consisting of small disjoint open intervals $I$ and $J$ around $a$ and $b$ in $S^1$. Denote by $H(I)$ and $H(J)$ the subgroups of $\homeoS$ supported on $I$ and $J$ respectively. We will show that $p^{-1}(U)$ is open by proving that 
    $$p^{-1}(U) = X - \left(\text{fix}(\rho(H(I)))\cup\text{fix}(\rho(H(J)))\right)$$
    Suppose $x\notin p^{-1}(U)$, then in particular for $p(x) = (\theta, \varphi)$, assume without loss of generality $\theta, \varphi \notin I$. Then by definition, every element of $H(I)$ fixes $(\theta, \varphi)$, so by equivariance, $x\in \text{fix}(\rho(H(I)))$. In particular, $x$ is not in the right hand side of the desired equality. Now, suppose $x\in p^{-1}(U)$, then in particular $p(x) = (\theta, \varphi)$ for some $\theta\in I, \varphi\in J$. Fix an equivariant homeomorphism $\varphi$ from the $\homeoS$ orbit of $x$ to $\pconf{2}$ such that $\psi(x) = (\theta, \varphi)$. Then, in particular we know $\rho(f)(x) = \psi^{-1}(f(\theta), f(\varphi))$ and thus if $f\in H(I) - G_\theta$ then $f\notin \text{Stab}(x)$. The contrapositive of this is then that $x\notin \text{fix}(\rho(H(I)))$. A symmetric argument will show that $x\notin \text{fix}(\rho(H(J)))$, so it is not in the union. Since fixed point sets are closed, their union is as well, thus the complement here is open. So this shows $p$ is open.

    Finally, we can demonstrate that for a point $(x,y)\in\conf{2}$, there is a homeomorphism from $F$ to $p^{-1}(x, y)$. Simply choose a homeomorphism $h\in\homeoS$ such that $h(b) = (x, y)$, then by equivariance of $p$, $\rho(h)$ restricts to a homeomorphism $F\to p^{-1}(x,y)$.
\end{proof}

Next, rather than belabor a proof that $p$ in fact gives a bundle structure, we will build a homeomorphism between $X$ and a particular generalized flat bundle over $\conf{2}$ that pulls its projection map back to $p$. 

\begin{rem}
    Suppose for some topological spaces $X, Y$ and $B$ such that $p: X\to B$ is a topological fiber bundle and $h: X\to Y$ is a homeomorphism. Suppose $q: Y\to B$ is a map such that the relevant triangle over $B$ commutes. Then $q:Y \to B$ is a fiber bundle and in particular $h$ is a bundle isomorphism.
\end{rem}

The relevant bundle is constructed as follows,
\begin{obs*}
    For convenience, assume that the $S^1$ coordinates of the basepoint are antipodal. There is an $\Z_2$ action on $\pconf{2}\times p^{-1}(b)$ given by $\tau\times \rho(r_\pi)|_{p^{-1}(b)}$ where $r_\pi$ is the half turn rotation and $\tau$ is the unique nontrivial deck transformation of $\pconf{2}\to\conf{2}$. The restriction $\rho(r_\pi)|_{p^{-1}(b)}$ is a homeomorphism from $p^{-1}(b)\to p^{-1}(b)$ since every orbit contains exactly two points of $p^{-1}(b)$ and $p$ is equivariant. This is a product of (fixed point-free) homeomorphisms, so this action is continuous and free, and $\pconf{2}\times p^{-1}(b)/\Z_2\to \pconf{2}\times p^{-1}(b)$ is a two-fold covering map.
\end{obs*}

We begin by building the map which we will demonstrate to be a homeomorphism. First we check that it is a continuous bijection:

\begin{prop}
    There is a continuous bijection from $X$ to $(\pconf{2}\times F)/\Z_2$ where $\Z_2$ acts on $\pconf{2}\times F$ by the involution which exchanges $\theta$ and $\phi$ coordinates on $\pconf{2}$ and swaps the all of the pairs of points in $F$ lying in the same orbit.
\end{prop}

\begin{proof}
    To start, we fix a basepoint $b\in \conf{2}$ and label the two choices of lifts in $\pconf{2}$ as $\tilde{b}_0$ and $\tilde{b}_1$ with the property that $\tilde{b}_0 = \tau(\tilde{b}_1)$ where $\tau:\pconf{2}\to\pconf{2}$ is the involution which swaps the two points on the circle.Then, if $x\in F=p^{-1}(b)$, by the orbit classification theorem, there are two choices of equivariant homeomorphism from the orbit of $x$ to $\pconf{2}$ with the standard action; namely, $\Phi_{0_x}$ and $\Phi_{1_x}$ which send $x$ to $\tilde{b}_0$ and $\tilde{b}_1$ respectively.
    We begin by constructing a map
    $$\hat{I}: \pconf{2}\times F \to X.$$
    Given $x=(\alpha, \beta)\in \pconf{2}$ and $y\in F$, then choose $f_x\in\homeoS$ such that $f_x*(\alpha, \beta) = \tilde{b}_0$. Then,
    $$\hat{I}((\alpha, \beta)\times y) \coloneqq \rho(f_x^{-1})(y)$$

    {\it $\hat{I}$ is well-defined.} Suppose $f_x$ and $g_x$ are two such maps. Then, $f_x\circ g_x^{-1}$ fixes $b$, thus also fixes $y$. Reworded, there is some $h\in \text{stab}(y)$ such that $f_x = h\circ g_x$. Then,
    $$\rho(f_x^{-1})(y) = \rho(g_x^{-1}h^{-1})(y) = \rho(g_x^{-1})(y)$$

    {\it $\hat{I}$ descends.} Let $(\alpha, \beta)\times y$ and $(\beta, \alpha)\times y'$ be two points which are identified by the $\Z_2$ action. Notice that the map $\Phi_0$ centered at $y$ must take $y'$ to $\tilde{b}_1$. Now choose a map $f_x\in\homeoS$ which moves $(\alpha, \beta)$ to $\tilde{b}_0$, and $f'_x$ which moves $(\beta, \alpha)$ to $\tilde{b}_0$. By working in $\Phi_0$ coordinates at $y$, we see that $\rho(f_x^{-1})(y) = \Phi_{0_y}^{-1}(f_x^{-1} * \tilde{b}_0)$. By construction, this is the same as $\Phi_{0_y}^{-1}(\alpha, \beta)$. By the same argument, $\rho(f'{_x}^{-1})(y') = \Phi_{0_{y'}}^{-1}(\beta, \alpha)$. Since the change of coordinates map is $\tau$, these are the same point. 

   {\it $I$ is bijective.} We refer to the induced map on the quotient as $I$. We wish to demonstrate that this map is a homeomorphism. Surjectivity is clear since every point in $X$ belongs to some orbit of a point in $p^{-1}(b)$. To see injectivity, suppose $(\theta, \varphi)\times y$ and $(\theta', \varphi')\times y'$ map to the same point under $I$. First note that, trivially, $y$ and $y'$ must be points of $p^-1(b)$ which lie in the same orbit in $X$. This then proceeds by an identical argument to the argument that the map descends, ultimately showing that if $y$ and $y'$ are the same point, we must have $(\theta, \varphi) = (\theta', \varphi')$ and if $y$ and $y'$ are distinct, $(\theta, \varphi) = \tau(\theta', \varphi')$.\improve{could write something slightly slicker with change of coordinates map here}\

   {\it $I$ is continuous.} Finally, we will check sequential continuity. Suppose $(m_i, y_i)$ converges to $(m, y)$ in $(\pconf{2}\times F)/\Z_2$. Then, we are free to choose a sequence $\{f_{m_i}\} \subset \homeoS$ converging to $f_m$ a representative function for producing $I(m, y)$. Then, by the continuity of the $\homeoS$ action $\{\rho(f_{m_i})\}$ is a sequence converging to $\rho(f_m)$ in $\text{Homeo}_0(X)$. In particular, since $X$ is a manifold, convergence in $\text{Homeo}_0(X)$ implies pointwise convergence, so $$I(m_i, y_i) = \rho(f_{m_i}^{-1})(y_i) \to \rho(f_{m}^{-1})(y) = I(m, y).$$
\end{proof}

To complete the proof that this map is, in fact, a homeomorphism, we will show that it is proper. To do this, we will need the following lemma.

\begin{lem}\label{lem:cpct_subset_union}
    Suppose $M, \,N$ closed manifolds such that $\text{Homeo}_0(M)$ acts continously on $N$. Further, suppose $K$ is a compact subset of $N$ and $S$ is a compact subset of $\text{Homeo}_0(M)$, then $\rho(S)(K) \coloneqq \bigcup_{\sigma\in S} \rho(\sigma)(K)$ is compact.
\end{lem}

\begin{proof}
    Let $\{x_n\}$ be a sequnce of points in $\rho(S)(K)$ converging to $x$. Then denote by $\sigma_n$ a sequence of elements of $S$ such that $x_n\in\rho(\sigma_n)(K)$. Using the continuity of the action map, clearly $\rho(S)\subseteq\text{Homeo}_0(N)$ is compact. Since $\text{Homeo}_0(N)$ is metric, we can pass to a convergent subsequence of $\{\rho(\sigma_n)\}$. In particular, since $N$ is closed, convergence of a sequence in $\text{Homeo}_0(N)$ implies uniform convergence when considered as a sequence of functions from $N$ to $N$. Since $\rho(\sigma_n)$ converges to some $\rho(\sigma)$ uniformly, and inversion is continuous in $\text{Homeo}_0(N)$, $\rho(\sigma_n^{-1})$ converges uniformly to $\rho(\sigma^{-1})$. Then, by the general metric space remark \ref{rem:metric_fnxn}, $\rho(\sigma_n^{-1})(x_n)$ converges to $\rho(\sigma^{-1})(x)$. But, by choice, $\rho(\sigma_n^{-1})(x_n)$ is a sequence in $K$, so in particular, $\rho(\sigma^{-1})(x)\in K$ since $K$ is closed by the Hausdorffness of $N$. But, this means that $x\in\rho(\sigma)(K)\subseteq\rho(S)(K)$; thus, $\rho(S)(K)$ is closed, and in particular, compact.
\end{proof}

For completeness, the following is a brief presentation of the metric space fact used in the above proof

\begin{rem}\label{rem:metric_fnxn}
    Suppose $X,\,Y$ are metric space, $f_n: X \to Y$ a sequence of continuous functions converging uniformly to $f$ and $\{x_n\}$ is a sequence converging to $x$, then $f_n(x_n)$ converges to $f(x)$
\end{rem}
\begin{proof}
    Let $\varepsilon>0$. By uniform convergence of $\{f_n\}$, there is some $N$ such that for all $n\geq N$ we have $\text{d}(f_n(x_n), f(x_n)) \leq \varepsilon/2$ for all $n\geq N$. Since $f$ is continuous and $x_n$ converges to $x$, there is some other $N'$ such that for all $n\geq N'$ we have $\text{d}(f(x_n), f(x)) \leq \varepsilon/2$. By taking $n\geq \text{max}\{N,\, N'\}$, and applying the triangle inequality, we arrive at $\text{d}(f_n(x_n), f(x)) \leq \varepsilon$ as desired.
\end{proof}.


We will complete the proof that $I$ is a homeomorphism, by showing that $I$ is {\it proper} since a continuous, proper bijection is a homeomorphism. Recall,
\begin{defn}
    Let $X$ and $Y$ be metric spaces. A sequence $\{x_n\}$ {\it diverges to infinity} if every compact set $K$ in $X$ contains at most finitely many points of $\{x_n\}$.
    
    A map $f:X\to Y$ is {\it proper} if $\{x_n\}$ diverges to infinity implies $\{f(x_n)\}$ diverges to infinity. 
\end{defn}

Finally,
\begin{prop}
    The map $I:(\pconf{2}\times F)/\Z_2\to X$ is proper.
\end{prop}
\begin{proof}
    Let $(x_n)$ be a sequence in $(\pconf{2}\times F)/\Z_2$ diverging to infinity. We consider the sequence $(I(x_n))\subset X$. Let $K\subsetneq X$ be some compact set. We wish to show that $K$ may contain only finitely many points of $(I(x_n))$. First, consider $\{p(I(x_n))\}\cap p(K)$. If this collection were finite, we would be finished as $K$ could then only contain finitely many of $(I(x_n))$, so suppose without loss of generality it is infinite. Denote by $(y_n)$ the subsequence of $(x_n)$ such that $p(I(y_n))\in p(K)$. Note that $p$ is continuous, so in particular $p(K)$ is compact; furthermore, since $\pi:\pconf{2}\to \conf{2}$ is a finite-sheeted covering, it is proper, so $\pi^{-1}(p(K)) \subsetneq \pconf{2}$ is compact. By the definition of the map $p$, if $p(I(x_n))$ is in $p(K)$, then there is a lift $\tilde{x}_n$ such that $\text{pr}_{\pconf{2}}(\tilde{x_n})$ is in $\pi^{-1}(p(K))$. Thus there exist lifts $(\tilde{y}_n)$ such that $\pi{-1}(p(K))$ contains $(\text{pr}_{\pconf{2}}(\tilde{y}_n))$. If there were some compact $K'$ containing infinitely many of $(\text{pr}_{F}(\tilde{y}_n))$ for any choice of lifts, then the image of $\pi^{-1}(p(K))\times K'$ under the $\Z_2$ quotient is a compact set containing infinitely many points of $(y_n)\subseteq(x_n)$, but $(x_n)$ diverges to infinity, so no such compact set can exist. Thus, there is no such $K'$ and $\text{pr}_F(\tilde{y}_n)$ must diverge to infinity for any choices of lifts. 

    The goal for what follows is to construct a large compact set $\hat{K}$ containing $K$ with the property that $I(y_n)\in \hat{K}$ if and only if $\text{pr}_F(\tilde{y}_n)\in \hat{K}\cap p^{-1}(b_0)$ as such a $\hat{K}$ may contain only finitely many of $(I(y_n))$. Consider the set $\mathcal{P}_N*SO(2)\subset \homeoS$ defined as:
    $$\mathcal{P}_N*SO(2)\coloneqq \left\{ P_t R \;\vert \;P_t = \begin{pmatrix} 1 & t\\ 0 & 1\\
    \end{pmatrix}\, t\in[-N, N], R\in SO(2)\right\}$$
    Where $SO(2)$ is the rotation subgroup and $P_t$ is considered as an element of $PSL(2;\R)$. Note that $\mathcal{P}_N * SO(2)$ is a group product of two continuous families in $\homeoS$ which intersect at the identity, so it is homeomorphic to their topological product, i.e. a closed annulus; thus, $\mathcal{P}_N*SO(2)$ is compact for all $N\in\R$. Moreover, if $B\subset\conf{2}$ is compact, there is some $N\in \R$ with the property that for every $x\in B$, there is some element $f\in \mathcal{P}_N*SO(2)$ such that $f*x = b$. The procedure to construct this element is as follows. Since $B$ is compact, there is a uniform $\varepsilon$ such that for all $(\alpha, \beta)\in B$, $\lvert\beta - \alpha\rvert\ >\epsilon$. Suppose without loss of generality that $b_0 = (0,\theta)$ for some $\theta$. Then, let $\varepsilon' =\text{min}(\varepsilon,\, \theta,\, 1-\theta)$. Now, for $N=2*\cot(\pi \varepsilon')$, the element $P_N$ satisfies $P_N(\varepsilon') = 1-\varepsilon'$, so in particular, for any $\beta, \beta'\in[\varepsilon', 1-\varepsilon']$ there is some $t_{\beta\beta'}\in[-N, N]$ such that $P_{t_{\beta\beta'}}(\beta) = \beta'$. Thus, the element of $\mathcal{P}_N * SO(2)$ taking $(\alpha, \beta)$ to $(0, \theta)$ is $P_{t_{(\beta-\alpha)\theta}} R_{-\alpha}$. Moreover, since, by construction, $N$ was uniform over $B$, this is satisfied by the same $\mathcal{P}_N*SO(2)$.

    Now, fix an appropriate $N$ for $p(K)$, then by applying lemma \ref{lem:cpct_subset_union}, we get that
    $$\hat{K}\coloneqq \bigcup_{f\in \mathcal{P}_N * SO(2)} \rho(f)(K)$$
    is a compact set containing $K$. Moreover, $\hat{K}$ has the property that if $I(x)\in K$, then $\text{pr}_F(\tilde{x})\in \hat{K}\cap p^{-1}(b)$ which is a subset of the fiber $F$, since $F$ is defined to be $p^{-1}(b)$. This is easy to see since fixing an orbit $A$, by definition the $I^{-1}$ image of each point in $A$ has a lift whose $F$ coordinate is each of the points in $p^{-1}(b)\cap A$. At this point, we are essentially finished though. Since $p{-1}(b)$ is closed, $\hat{K}\cap p^{-1}(b)$ is compact in $X$ and thus compact in $p{-1}(b)$. But, as observed earlier, any sequence $(\text{pr}_F(\tilde{y}_n))$ must diverge to infinity where $(\tilde{y}_n)$ is any sequence of lifts $(y_n)$, so in particular, $\hat{K}$ contains only finitely many $F$ projections of lifts of points in $(y_n)$, so $K$ contains only finitely many of the points of $(y_n)$ and thus $(x_n)$
\end{proof}

\section{2D Orbits Compactify Nicely}
As a small point of notation, recall the following definition from point set topology:
\begin{defn}
    The {\it frontier} of a set $X$, denoted $\text{fr}(X)$ is defined as $\cl{X}-X$
\end{defn}

When $X$ is a 2-dimensional orbit, we use the following notation for convenience.

\begin{defn}
    Let $A$ be an annulus orbit, fix a model annulus $f:A\to S^1\times I$. Then, denote by $\text{fr}_+(A)$ the set of frontier points which can be accumulated by sequences with increasing $I$ coordinate and $\text{fr}_-(A)$ analogously. Note that $\text{fr}_+(A)\cup\text{fr}_-(A) = \text{fr}(A)$, but they are not necessarily distinct. We can compatibly label subsets of the frontier of a $G_0$ invariant interval in an annulus $A$. For notational consistency, when $X$ is a M\"{o}bius band orbit, let $\text{fr}_+(X) = \text{fr}_-(X) = \text{fr}(X)$.
\end{defn}

For the following two proofs, we want to make use of a kind of canonical coordinates on interval orbits of $G_0$.

\begin{rem}\cite{mann-chen}
    Suppose $A$ is an annulus orbit of $\homeoS$ and $I\subset A$ is an interval orbit of $G_0$. Then, we can define two homeomorphisms $\varphi, \psi:(0,1) \to I$ by $\varphi(\theta) = \text{fix}(G_0\cap G_\theta)$ and $\psi(\theta) = \text{fix}(G_0\cap G_{-\theta})$. One of these maps will send monotonic sequences that converge to $1$ to sequences which converge to something in $\text{fr}_+(I)$, we will denote the inverse of this map as $f_I$, which we will refer to as the {\it canonical homeomorphism between $I$ and $S^1 - \{0\}$}. Note, the same construction works for the one interval orbit in a M\"{o}bius band orbit.
\end{rem}

The following results are true for 2-dimensional orbits of any $\homeoS$ action without fixed points on a closed 3-manifold.

\begin{prop}
    Suppose $X$ is an 2-dimensional orbit, then $\text{fr}(X)$ consists of 1-dimensional orbits.
\end{prop}

\begin{proof}
    Trivially, $\text{fr}(X)$ contains no 3-dimensional orbit.

    Suppose toward a contradiction that $\text{fr}(X)$ contains some 2-dimensional orbit $X'$. Now, consider the restriction of the $\homeoS$ action to the point stabilizer subgroup $G_0$. There is some interval orbit $I$ in $X$. Since $\pi_{SO(2)}$ is continuous and closed, $$\pi_{SO(2)}(\bar{I}) = \overline{\pi_{SO(2)}(I)} = \overline{\pi_{SO(2)}(A)} = \pi_{SO(2)}(\bar{A}).$$ In particular, this means that $\text{fr}(I)$ must nontrivially intersect $X'$. The closure of every $G_0$ orbit in $X'$ contains at least one interval orbit $I'$, so $I'\subseteq \text{fr}(I)$. In particular, there is some sequence of points $\{x_n\}\subset I$ converging to a point $x\in I'$.

    From the remark, there are continuous bijections $f:I\to S^1-\{0\}$ and $g:I'\to S^1-\{0\}$ given by mapping the unique fixed point of $G_\theta\cap G_0$ on $I$ to $\theta\in S^1 - \{0\}$ and similarly for $I'$. Then, up to a subsequence, $\{f(x_n)\}$ is a monotonic sequence converging to 0. So we can choose a monotonic element $\phi\in G_0$ satisfying $\phi(f(x_k)) = f(x_{k+1})$. In particular, we must have that $\{\rho(\phi)( x_n)\}$ converges to $x$, but by monotonicity, $\rho(\phi)(x) \neq x$, violating continuity, which is a contradiction. Thus $\text{fr}(A)\cap X = \emptyset$
\end{proof}

Note that, in particular if $X$ is an annulus orbit in a component foliated by annuli, this implies that $\text{fr}(X)$ is a subset of the frontier of the component.

We can in fact say something stronger.

\begin{prop}
    Suppose $X$ is a 2-dimensional orbit. Then, $\text{fr}_+(X)$ and $\text{fr}_-(X)$ are (not necessarily distinct) one-dimensional orbits.
\end{prop}

\begin{proof}
    Suppose that $\sigma$ and $\sigma'$ are invariant circles in (WLOG) $\text{fr}_+(X)$ with respective $G_0$ fixed points $x$ and $x'$. Let $I$ be a $G_0$-invariant interval in $X$. By the same reasoning as the previous result, $x, x'\in \text{fr}(I)$, so let $\{a_k\}$ and $\{b_k\}$ be sequences in $I$ reperesenting $x$ and $x'$. Since both sequences are monotonic in $I$, there is a homeomorphism $f\in G_0$ such that $\rho(f)(a_k) = b_k$. By continuity, this implies that $\{b_k\}$ converges to $\rho(f)(x)$, which is $x$, since $x\in\text{fix}(G_0)$. Thus, $x=x'$; morevoer, $\sigma = \sigma'$ since every fixed point of $G_0$ is part of a unique 1-dimensional orbit of $\homeoS$.
\end{proof}

This result can be rephrased in a slightly more concrete way.

\begin{cor}
    When $X$ is an annulus orbit, $\bar{X}$ is either an invariant closed annulus with boundary circles $\text{fr}_+(A)$ and $\text{fr}_-(A)$ or an invariant $T^2$ when $\text{fr}_-(A) = \text{fr}_+(A)$.

    When $X$ is a M\"{o}bius band orbit, $\bar{X}$ is an invariant closed M\"{o}bius band, with invariant boundary circle $\text{fr}(X)$.
\end{cor}

\listoftodos[Notes]
\end{document}