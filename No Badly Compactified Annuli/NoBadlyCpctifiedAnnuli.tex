\documentclass[10pt, oneside]{article} 
\usepackage{mathtools, amsmath, amsthm, amssymb, calrsfs, wasysym, verbatim, bbm, color, graphics, geometry, xargs}
\usepackage[pdftex,dvipsnames]{xcolor}

\usepackage[colorinlistoftodos,prependcaption,textsize=tiny]{todonotes}

\newcommandx{\fix}[2][1=]{\todo[linecolor=red,backgroundcolor=red!25,bordercolor=red,#1]{#2}}
\newcommandx{\improve}[2][1=]{\todo[linecolor=blue,backgroundcolor=blue!25,bordercolor=blue,#1]{#2}}
\newcommandx{\note}[2][1=]{\todo[linecolor=OliveGreen,backgroundcolor=OliveGreen!25,bordercolor=OliveGreen,#1]{#2}}
\newcommandx{\unsure}[2][1=]{\todo[linecolor=Plum,backgroundcolor=Plum!25,bordercolor=Plum,#1]{#2}}

\geometry{tmargin=.75in, bmargin=1in, lmargin=1in, rmargin = 1in}  

%% NOTATION
%%%%%
\newcommand{\R}{\mathbb{R}}
\newcommand{\C}{\mathbb{C}}
\newcommand{\Z}{\mathbb{Z}}
\newcommand{\N}{\mathbb{N}}
\newcommand{\Q}{\mathbb{Q}}
\newcommand{\Cdot}{\boldsymbol{\cdot}}
\newcommand{\SO}{\text{SO}(2)}
\newcommand{\homeoS}{\text{Homeo}_0(S^1)}
\newcommand{\cl}[1]{\overline{#1}}

%% CUSTOM THEOREMS
%%%%%
\newtheorem{thm}{Theorem}
\newtheorem*{thm*}{Theorem}
\theoremstyle{definition}
\newtheorem{defn}{Definition}
\newtheorem{conv}{Convention}
\newtheorem{conj}{Conjecture}
\newtheorem{rem}{Remark}
\newtheorem*{obs*}{Observation}
\newtheorem{lem}{Lemma}
\newtheorem{cor}{Corollary}
\newtheorem{prop}{Proposition}
\theoremstyle{definition}
\newtheorem*{prog*}{Program}

% \usepackage{biblatex}
% \addbibresource{InvariantAnnuli.bib}

\title{Open Regions Foliated by Annulus Orbits}
\author{Hazel Brenner}
\date{Summer 2023}

\begin{document}

\maketitle

\section{Introduction}
The following is a result about the structure of compact three-manifolds admitting fixed point-free $\homeoS$ actions having only one dimensional and annulus orbits. A fact we will make repeated use of to prove these results is the following:

\begin{rem}
    Let $\pi_{SO(2)}$ denote the projection along the orbits of the rotation subgroup to the base orbifold. Then, since $M^3$ and the base orbifold are compact Hausdorff spaces, by a classical fact from point-set topology, $\pi_{SO(2)}$ is a closed map.
\end{rem}

\section{Global structure}
Consider the subset of $M$ consisting of 1-dimensional orbits. The set $\text{fix}(G_0)$ is closed, and $\pi_{SO(2)}$ is a closed map, so $\pi_{SO(2)}(\text{fix}(G_0))$ is closed in the base surface. By continuity then $\pi_{SO(2)}^{-1}(\pi_{SO(2)}(\text{fix}(G_0)))$,which is the union of all of the one-dimensional orbits, is closed.
The complement is then an open subset of $M$ which decomposes into annulus orbits. It is natural to ask about the structure of this open region. Below are two ipmortant structural results:

\section{Foliation by Annuli}

\begin{prop}
    Suppose $X$ is a component of the open subset of $M^3$ in which all orbits are annuli. Then the decomposition of $X$ into orbits is a $\text{C}^0$ foliation by annuli.
\end{prop}
\begin{proof}[Proof sketch]\improve{flesh this out with details}
    Take a short curve segment through $x$, build two one paramater families of homeos, giving local product structure (do this explicitly, use flow of a bump vec field on circle).
\end{proof}

\section{Annuli Compactify Nicely}
As a small point of notation, recall the following definition from point set topology:
\begin{defn}
    The {\it frontier} of a set $X$, denoted $\text{fr}(X)$ is defined as $\bar{X}-X$
\end{defn}

When $X$ is an annulus orbit, we use the following notation for convenience.

\begin{defn}
    Let $A$ be an annulus orbit, fix a model annulus $f:A\to S^1\times I$. Then, denote by $\text{fr}_+(A)$ the set of frontier points which can be accumulated by sequences with increasing $I$ coordinate and $\text{fr}_-(A)$ analogously. Note that $\text{fr}_+(A)\cup\text{fr}_-(A) = \text{fr}(A)$, but they are not necessarily distinct. We can compatibly label subsets of the frontier of a $G_0$ invariant interval in an annulus $A$
\end{defn}

For the following two proofs, we want to make use of a kind of canonical coordinates on interval orbits of $G_0$.

\begin{rem}
    Suppose $A$ is an annulus orbit of $\homeoS$ and $I\subset A$ is an interval orbit of $G_0$. Then, we can define two homeomorphisms $\varphi, \psi:(0,1) \to I$ by $\varphi(\theta) = \text{fix}(G_0\cap G_\theta)$ and $\psi(\theta) = \text{fix}(G_0\cap G_{-\theta})$. One of these maps will send monotonic sequences that converge to $1$ to sequences which converge to something in $\text{fr}_+(I)$, we will denote the inverse of this map as $f_I$, which we will refer to as the {\it canonical homeomorphism between $I$ and $S^1 - \{0\}$}
\end{rem}

\begin{prop}
    Suppose $A$ is an annulus orbit in a component $X$ of the open region foliated by annulus orbits. Then $\text{fr}(A)\subset \text{fr}(X)$
\end{prop}

\begin{proof}
    Suppose toward a contradiction that $\text{fr}(A)\cap X\neq \emptyset$. Then, since every fiber in $X$ is contained in some annulus orbit and $\bar{A}$ must be invariant, there is some invariant annulus $A'\subset \text{fr}(A)$. Now, consider the restriction of the $\homeoS$ action to the action of a point stabilizer subgroup $G_0$. Let $I$ be a $G_0$ invariant interval in $A$. Since $\pi_{SO(2)}$ is continuous and closed, $$\pi_{SO(2)}(\bar{I}) = \overline{\pi_{SO(2)}(I)} = \overline{\pi_{SO(2)}(A)} = \pi_{SO(2)}(\bar{A}).$$ This implies that $\text{fr}(I)$ intersects every fiber of $\text{fr}(A)$ and in particular, every fiber of $A'$. Since $\bar{I}$ is closed and invariant, $\text{fr}(I)$ must be a union of $G_0$ orbits in $A'$ containing at least one interval orbit $I'$ (the disk orbits are not closed). Thus, there is some sequence of points $\{x_n\}\subset I$ converging to a point $x\in I'$.


    From \cite{mann-chen}, there are continuous bijections $f:I\to S^1-\{0\}$ and $g:I'\to S^1-\{0\}$ given by mapping the unique fixed point of $G_\theta\cap G_0$ on $I$ to $\theta\in S^1 - \{0\}$ and similarly for $I'$. Then, up to a subsequence, $\{f(x_n)\}$ is a monotonic sequence converging to 0. So we can choose a monotonic element $\phi\in G_0$ satisfying $\phi(f(x_k)) = f(x_{k+1})$. In particular, we must have that $\{\rho(\phi)( x_n)\}$ converges to $x$, but by monotonicity, $\rho(\phi)(x) \neq x$, violating continuity, which is a contradiction. Thus $\text{fr}(A)\cap X = \emptyset$
\end{proof}

Note that, in particular, $\text{fr}(X)$ consists entirely of one-dimensional orbits, so $\text{fr}(A)$ must as well. A piece of notation which will be useful for the following,



In fact, we can say something quite strong about these sets.

\begin{prop}
    Suppose $A$ is an annulus orbit. Then, $\text{fr}_+(A)$ and $\text{fr}_-(A)$ are (not necessarily distinct) one-dimensional orbits.
\end{prop}

\begin{proof}
    Suppose that $\sigma$ and $\sigma'$ are invariant circles in $\text{fr}_+(A)$ with respective $G_0$ fixed points $x$ and $x'$. Let $I$ be a $G_0$-invariant interval in $A$. By the same reasoning as the previous result, $x, x'\in \text{fr}(I)$, so let $\{a_k\}$ and $\{b_k\}$ be sequences in $I$ reperesenting $x$ and $x'$. Since both sequences are monotonic in $I$, there is a homeomorphism $f\in G_0$ such that $\rho(f)(a_k) = b_k$. By continuity, this implies that $\{b_k\}$ converges to $\rho(f)(x)$, which is $x$, since $x\in\text{fix}(G_0)$. Thus, $x=x'$; morevoer, $\sigma = \sigma'$ since every fixed point of $G_0$ is part of a unique 1-dimensional orbit of $\homeoS$.
\end{proof}

This result can be rephrased in a slightly more concrete way.

\begin{cor}
    The invariant subset $\bar{A}$ is either an invariant closed annulus with boundary circles $\text{fr}_+(A)$ and $\text{fr}_-(A)$ or an invariant $T^2$ when $\text{fr}_-(A) = \text{fr}_+(A)$.
\end{cor}
\textcolor{blue}{Next steps: show that $G_0$ invariant intervals must have frontier = $\text{fix}(G_0)\cap\text{fr}(A)$?}

\listoftodos[Notes]
\end{document}