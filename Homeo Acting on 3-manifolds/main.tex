\documentclass[10pt, oneside]{article}
\usepackage[pdftex,dvipsnames, table, xcdraw]{xcolor}
\usepackage{mathtools,mathrsfs, amsmath, amsthm, amssymb, wasysym, verbatim, bbm, graphics, geometry, xargs, hyperref, svg, float, hyperref}
\usepackage{thm-restate}
\usepackage{relsize}
\usepackage[nameinlink, capitalize]{cleveref}
\usepackage[shortlabels]{enumitem}
\usepackage[colorinlistoftodos,prependcaption,textsize=tiny]{todonotes}
\usepackage{multirow}

\usepackage[
backend=biber,
style=numeric,
sorting=ynt,
]{biblatex}
\addbibresource{main.bib}

\newcommandx{\fix}[2][1=]{\todo[linecolor=red,backgroundcolor=red!25,bordercolor=red,#1]{#2}}
\newcommandx{\improve}[2][1=]{\todo[linecolor=blue,backgroundcolor=blue!25,bordercolor=blue,#1]{#2}}
\newcommandx{\wantfigure}[2][1=]{\todo[linecolor=OliveGreen,backgroundcolor=GreenYellow!25,bordercolor=OliveGreen,#1]{#2}}
\newcommandx{\add}[2][1=]{\todo[linecolor=Plum,backgroundcolor=Plum!25,bordercolor=Plum,#1]{#2}}


\geometry{tmargin=.75in, bmargin=1in, lmargin=1in, rmargin = 1in}

%% NOTATION
%%%%%
\newcommand{\RP}{\mathbb{RP}}
\newcommand{\R}{\mathbb{R}}
\newcommand{\C}{\mathbb{C}}
\newcommand{\Z}{\mathbb{Z}}
\newcommand{\N}{\mathbb{N}}
\newcommand{\Q}{\mathbb{Q}}
\newcommand{\D}{\mathbb{D}}
\newcommand{\Cdot}{\boldsymbol{\cdot}}
\newcommand{\SO}[1][2]{\text{SO}(#1)}
\newcommand{\homeo}[1][S^1]{\text{Homeo}_0(#1)}
\newcommand{\diffr}[1][M]{\text{Diff}^r(#1)}
\newcommand{\cl}[1]{\overline{#1}}
\newcommand{\conf}[2][S^1]{\text{Conf}_{#2}(#1)}
\newcommand{\pconf}[2][S^1]{\text{PConf}_{#2}(#1)}
\newcommand{\set}{{\{\cdot\}}}
\newcommand{\stab}[1]{\text{Stab}(#1)}
\newcommand{\pstab}[1]{\text{Stab}_0(#1)}
\newcommand{\maxcov}{C_{b_\set}}
\newcommand{\SL}[1][2]{\text{SL}(#1, \R)}
\newcommand{\PSL}[1][2]{\text{PSL}(#1, \R)}
\newcommand{\klim}[1]{\text{Lim}\,#1}
\newcommand{\dimn}[1]{\text{dim}(#1)}
\newcommand{\li}[1]{\text{Li}\,#1}
\newcommand{\ls}[1]{\text{Ls}\,#1}
\newcommand{\fr}[1]{\text{fr}(#1)}

\newcommand{\titlesafehomeo}{\texorpdfstring{$\homeo$}{Homeo\_0(S\string^ 1)}}
\newcommand{\titlesafehomeoM}{\texorpdfstring{$\homeo[M]\,$}{Homeo\_0(M)}}


%% CUSTOM THEOREMS
%%%%%
\newtheorem{thm}{Theorem}[section]
\newtheorem*{thm*}{Theorem}
\theoremstyle{definition}
\newtheorem{defn}{Definition}[section]
\newtheorem{eg}{Example}[section]
\newtheorem{conv}{Convention}[section]
\newtheorem{conj}{Conjecture}[section]
\newtheorem{rem}{Remark}[section]
\newtheorem*{obs*}{Observation}
\newtheorem*{construction*}{Construction}
\newtheorem{lem}{Lemma}[section]
\newtheorem*{lem*}{Lemma}
\newtheorem{cor}{Corollary}[section]
\newtheorem{fact}{Fact}
\newtheorem*{fact*}{Fact}
\newtheorem{prop}{Proposition}[section]
\newtheorem*{prop*}{Proposition}
\newtheorem*{clm*}{Claim}
\newtheorem*{note*}{Note}
\theoremstyle{definition}
\newtheorem*{prog*}{Program}
\newtheorem{question}{Question}[section]

\title{Towards a Complete Description of $\homeo$ Actions on Closed 3-manifolds}
\author{Hazel Brenner}

\begin{document}
\maketitle
\listoftodos
\tableofcontents

\section{Introduction}\label{sec:intro}

$C^r$ actions of compact Lie groups on manifolds satisfy strong local and global structural results, including but certainly not limited to slice theorems (Section 2.3 of \cite{duistermaat:LieGroups}), orbit-type stratification (Section 2.9 of \cite{duistermaat:LieGroups}), and complete classifications up to equivariant (diffeo)homeomorphism in various restricted settings. Intuitively, a local slice to a Lie group action is a $\stab{x}$-invariant submanifold through a point $x$ such that $G\cdot S$ is an open neighborhood of $G\cdot x$ which is $G$-isomorphic to a particular twisted $S$ bundle over $G/\stab{x}\cong G\cdot x$ (called a tube about $x$). Slice theorems guarantee certain settings in which slices exist; e.g, when $G$ is compact and $M$ is completely regular, slices exist through every point (consequence of Theorem 5.4 in \cite{bredon:IntroductionCompact}). Conjugacy of isotropy subgroups of orbits form an equivalence relation on the set of orbits whose equivalence classes are called orbit types. When $G$ is a Lie group acting properly on a manifold $M$, the orbit types are embedded submanifolds and moreover, they stratify $M$ (if the reader is unfamiliar with stratified spaces, an adequate model is the decomposition of a cell complex into cells of a fixed dimension.) These results are fundamental tools to study proper Lie group actions on manifolds, and are the starting point of Raymond and Orlik's classifications of $\SO$ actions on closed 3-manifolds and $T^2$ actions on simply connected 4-manifold\cite{orlik:ActionsSO2,orlik:ActionsTorusI, orlik:ActionsTorusII}. We offer the weak tube theorem and weak orbit type stratification as answers to these classical results. Details of the bundle structure and the $\homeo[M]$ action are contained in \cref{thm:precise-orbit-bundle}.
\begin{restatable}[Weak tube theorem]{thm}{orbitbundle}
    \label{thm:orbit-bundle}
        Suppose $\homeo[M]$ acts on a manifold $N$ with all orbits having fixed dimension $n*\dimn{M}$ for some $n\geq 1$. Then, $N$ must have the structure of a generalized flat bundle over $M$; moreover, the orbits form a foliation of $N$ transverse to the fibers.
    \end{restatable}
\begin{restatable}[Weak orbit type stratification]{thm}{weakotstratif}
\label{thm:weak-o-t-strat}
    Suppose $\homeo$ acts on closed 3-manifold $M$ with no global fixed points. Every orbit in $M_{(2)} - \cl{M_{(3)}}$ has a $\homeo$-invariant tubular neighborhood in the sense of \cref{thm:orbit-bundle} which is isomorphic to one of the examples in the top row of \cref{table:orbit-bundles}. Every orbit in $M_{(1)} - \cl{M_{(2)}\cup M_{(3)}}$ has an invariant neighborhood isomorphic to $\Sigma \times S^1$ where $\Sigma$ is an open surface of finite genus. Moreover, finitely many components of $M_{(2)} - \cl{M_{(3)}}$ are of exceptional type and finitely many components of $M_{(1)} - \cl{M_{(2)}\cup M_{(3)}}$ have $\Sigma$ with nonzero genus.
\end{restatable}

If we extend our focus from Lie groups to homeomorphism and diffeomorphism groups the picture is much less well-understood, but a surprising amount of similar results still hold. In particular, Militon \cite{militon:ActionsGroup} classifies continuous actions of $\homeo$ on the 2-torus and the closed annulus. Militon's classification decomposes the annulus (or torus) into an invariant lamination by concentric circles and gives a complete description of the action on the complement which is a disjoint union of open annulus orbits. Chen and Mann \cite{chen:StructureTheorems} go on to give a complete classification of spaces with transitive $\diffr$ and $\homeo[M]$ actions and apply this result to extend Militon's classification to all actions of $\homeo$ on compact surfaces. In particular, the extended classification shows that the global fixed points of a $\homeo$ action on a compact surface are exactly those of the standard rotation group action on that surface. That is, the disk has a unique fixed point, the sphere has two, etc. Away from fixed points, the behavior is ultimately shown to be exactly that of Militon's classification for the closed annulus and the torus. This paper describes the extent to which similar results can be achieved for $\homeo$ actions on closed 3-manifolds.

It is valuable to situate our problem in the context of the dimensions of both the manifold $M$ whose homeomorphism group is acting and the manifold $N$ being acted upon. A key restriction proved by Chen and Mann \cite{chen:StructureTheorems} is that homeomorphism groups of manifolds cannot act non-trivially on strictly lower-dimensional manifolds (Theorem 1.4). Moreover, when the dimension $M$ is large relative to the dimension of $N$ (specifically, $\dimn{N} < 2*\dimn{M}$), $N$ must be a topological generalized flat (or equivalently, foliated) bundle over $M$. Combining these results in the special case of $\dimn{N} = 3$ yields the following:
\begin{note*}
    Suppose $\homeo[M]$ acts on $N$ nontrivially, then $\dimn{M}\leq 3$. If $M$ is a 3-manifold, then $N$ is a disjoint union of covers of $M$. If $M$ is a surface, then $N$ is a generalized flat bundle over $M$.
\end{note*}
\noindent Thus, if one is interested in studying actions of homeomorphism groups on 3-manifolds, the only remaining case to be considered is that in which $M$ is a closed 1-manifold, i.e. $M\cong S^1$.

In a broad sense, the problem of classifying actions of a fixed topological group on spaces of a fixed dimension can be divided into the orbit classification problem and the orbit gluing problem. The orbit classification problem for a fixed Lie group is recognizable as the problem of classifying homogeneous spaces (or equivalently its closed subroups). The orbit gluing problem is addressed by slice theorems and the orbit-type stratification. For an example of this program in use, see Ghys's classification of $C^r$ actions of $\SL[n]$ on once-punctured $\R^m$\cite{cairns:LocalLinearization} or Orlik and Raymond's classification of continuous $\SO$ actions on closed 3-manifolds\cite{orlik:ActionsSO2}. For $\homeo[M]$ and $\diffr$, Chen and Mann completely resolve the orbit classification problem and solve the orbit gluing problem for particular restrictions on dimension. In this paper, we treat the orbit gluing problem when $M= S^1$ and $N$ is a closed three manifold and separately when all orbits are assumed to be the same dimension but no restrictions are placed on $M$ and $N$.

In addition to the weak tube theorem \cref{thm:orbit-bundle} and weak orbit type stratification \cref{thm:weak-o-t-strat}, we provide a description of the global topology of a closed 3-manifold admitting an action of $\homeo$.
\begin{restatable}{thm}{thmglobaltop}
    \label{thm:3mfld-global-top}
    Suppose $\homeo$ acts on closed 3-manifold $M$ without global fixed points. $M$ is an $\SO[2]$-manifold with exceptional orbits of multiplicity two or three lying in $\homeo$ orbits of type $\conf[S^1]{2}$ and $\conf[S^1]{3}$. Special exceptional orbits have open non-orientable $\homeo$ invariant tubular neighborhoods homeomorphic to $M\ddot{o}b\times S^1$. In particular, when $M$ is orientable, it is Seifert-fibered.
\end{restatable}



% In \cref{sec:orbit-bundle-lemma} we suppose that, for $M$ some compact manifold, $\homeo$ acts on any manifold $N$ such that all orbits are of a fixed dimension $n*\dimn(M)$. Under these assumptions, we prove that $N$ must be equivariantly homeomorphic a particular foliated bundle over $\conf[M]{n}$ --- the configuration space of $n$ unlabeled points in $M$. This echoes Theorem 1.4 in \cite{chen_structure_2023} which outlines dimension restrictions on $M$ and $N$ rather than the orbits, which yield the same result.

% In \cref{sec:homeos1}, we fix $M = S^1$ and $N$ any closed 3-manifold. A full classification of $\homeo[S^1]$ actions on compact surfaces was begun in \cite{militon_actions_2016} and completed in \cite{chen_structure_2023}: a natural question is whether such a classification is possible one dimension higher. While we do not present a full classification, we provide several strong structural results and a conjectural path to complete classification. The primary results of the paper are \cref{lem:orbit-bundle}, \cref{thm:main-structure-thm} and \cref{thm:2d-orbit-cpctification}.

% For the remainder of this section, we will provide a brief primer on the notation and results which will be foundational to the rest of the paper and conclude with a few basic examples and families of actions of $\homeo$ on closed 3-manifolds.

\section{Background}\label{sec:background}
This section serves to recall definitions and introduce our conventions for notions which will be continually used throughout the paper. \cref{subsec:seifprelims} deals with Seifert-fibered spaces and 3-manifolds with faithful $\SO$ actions. \cref{subsec:homeoprelims} introduces notation and results around homeomorphism groups, group actions and configuration spaces. At the end of the section, we classify the orbits of $\homeo$ actions on 3-manifolds in \cref{prop:orbit-computation} as a direct application of the Chen-Mann orbit classification theorem \cite{chen:StructureTheorems}.

\subsection{Seifert-fibered and \texorpdfstring{$S^1$}{S\^string 1}-manifold preliminaries}\label{subsec:seifprelims}

As mentioned previously, all orientable closed 3-manifolds admitting $\homeo$ actions are Seifert-fibered. Non-orientable closed 3-manifolds admitting $\homeo$ actions necessarily admit an faithful $\SO$ action (but not necessarily a Seifert-fibering). We will briefly provide the relevant definitions and notation associated to both classes of manifolds.

\begin{defn}
    Let $(p, q)$ be two coprime integers with $p>0$, the $(p, q)$ standard fibered solid torus is $D^2\times [0, 1]/\psi$ where $\psi:D^2\times\{0\}\to D^2\times{1}$ is a rotation by angle $2\pi\frac{q}{p}$. 
\end{defn}

This definition forms the set of local models for tubular neighborhoods of fibers in a Seifert fibration. More precisely,

\begin{defn}
    A Seifert fibration of a closed 3-manifold $M$ is a decomposition of $M$ into a disjoint union of circles (called fibers) with each having a tubular neighborhood diffeomorphic to a stanadard fibered solid torus.
\end{defn}

Fibers with tubular neighborhoods with $p>1$ are said to be singular, while those with $p=1$ are said to be regular. The space of fibers of $M$ is a orbifold surface called the base orbifold with cone points of multiplicity $p$ corresponding to multiplicity $p$ singular fibers. There is a unique orientable Seifert-fibered 3-manifold whose base surface has underlying surface $\Sigma$ and whose singular fibers have invariants $(p_1, q_1),\dots,(p_n, q_n)$, which we will generally denote as $(\Sigma, (p_1, q_1),\dots,(p_i,q_i))$ \cite{martelli:IntroductionGeometric}.

To connect our setting to Seifert fibrations, we observe that there is a tight correspondence between faithful $\SO$ actions and Seifert fibrations.

\begin{rem}\label{rem:so2seifequiv}
    Let $M$ be a compact orientable 3-manifold. Seifert-fibrations of $M$ up to fiber-preserving homeomorphism are in one-to-one correspondence with faithful actions of the circle group $\SO$ on $M$ up to equivariant homeomorphism.
\end{rem}

In general, when it is clear that a given manifold is compact, orientable and Seifert-fibered, we will refer without comment to the associated $\SO$ action and vice versa for the Seifert-fibration associated to a given $\SO$ action. When $M$ is non-orientable, the situation is somewhat subtler -- all Seifert fibrations have an associated faithful $\SO$ action, but there are faithful $\SO$ actions with no associated Seifert fibration. These can be accounted for by allowing for the base orbifold to have mirror boundary. A fiber over a point in a mirror boundary will have tubular neighborhood $\SO[2]$-equivariantly homeomorphic to a solid Klein bottle with the analogue of the standard action by rotations on the solid torus.

\subsection{\titlesafehomeoM preliminaries}\label{subsec:homeoprelims}

Unless otherwise mentioned, all manifolds are assumed to be closed and equipped with some metric -- typically denoted $d$ without further comment. $\text{Homeo}(M)$ is the topological group of all homeomorphism of $M$ and $\homeo[M]$ is the connected component of the identity. It is a classical result that the compact-open topology on $\text{Homeo}(M)$ is the unique complete separable group topology\cite{kallman:UniquenessResults}. A convenient fact about the compact-open topology which we will use is that covergence of a sequence of homeomorphisms is equivalence to uniform convergence on compact subsets of M. The following is the fundamental definition of study:

\begin{defn}
    An abstract group homomoprhism $\rho: \homeo[M]\to\homeo[N]$ is said to be an action of $\homeo[M]$ on $N$. 
\end{defn}

By the automatic continuity property for homeomorphism groups\cite{mann:AutomaticContinuity}, all such actions are continuous. We will only consider actions of $\homeo[M]$, the identity component of $\text{Homeo}(M)$. Since the continuous image of a connected set is connected, any homomorphism from $\homeo[M] \to\text{Homeo}(N)$ will have image in $\homeo[N]$.

The simplest continuous action of $\homeo[M]$ is of course that on $M$ itself. A very useful property of the group is the following.

\begin{prop}\label{prop:lct}
    The action of $\homeo[M]$ on $M$ is {\it locally continuously transitive}. That is to say for every pair of points $x, y\in M$, there is a ball $B$ and a one-parameter subgroup $f_t\in \homeo[M]$ supported on $B$ such that $f_1(x) = y$.
\end{prop}

We will make frequent use of the subgroups of $\homeo$ which consist of homeomorphisms which fix a particular point on the circle. In the interest of avoiding ambiguity, we make the following definition.

\begin{defn}
     For $\theta\in S^1$, let $G_\theta\coloneqq\{f\in\homeo[S^1]\;\vert\; \theta\in\text{fix}(f)\}$.
     For $\theta_1,\dots,\theta_n\in S^1$, let $G_{\theta_1,\dots,\theta_n}\coloneqq \{f\in\homeo\;\vert\; \theta_1,\dots,\theta_n\in\text{fix}(f)\}$.
     For a general manifold $M$ and finite point set $X\subset M$, we write $\stab{X}\coloneqq\{f\in\homeo[M]\vert f(X) = X\}$, and denote by $\pstab{X}$ the connected component of the identity.
\end{defn}

In general, the pointwise stabilizer of $X$ is contained between $\stab{X}$ and $\pstab{X}$, but it is not true in general that it is equal to $\pstab{X}$. For example, consider the standard action of $\homeo[\D^2]$ on $\D^2$ and a finite point set $X\subset \D^2$. The quotient of the pointwise stabilizer of $X$ by $\pstab{X}$ is isomorphic to the pure braid group on $|X|$ strands. This is not to be confused with the stabilizer of a point or point set in $N$ under an action of $\homeo$ on $N$. When this is, in fact, what we are referring to, we will clearly clearly indicate it. 

Configuration spaces are central to the study of homeomorphism group actions. We adopt the following notational conventions for the spaces of configurations of labeled and unlabeled points.

\begin{defn}
    Given any $n$ and any topological space $X$, define the space $\pconf[M]{n} \coloneqq \{(x_i)\in X^n \,\vert\, \forall i\neq j,\, x_i\neq x_j\}$.
    There is a natural properly discontinuous, free action the symmetric group $S_n$ on $\pconf[M]{n}$ by permuting coordinates. Define $\conf[M]{n}\coloneqq \pconf[M]{n}/S_n$
\end{defn}

For all $n$, when $X$ is a topological manifold without boundary $\pconf[X]{n}$ and $\conf[X]{n}$ are as well. A small technicality arises when $X=S^1,\; \R$; namely, homeomorphisms of $S^1$ and $\R$ respectively preserve cyclic and linear orders of the coordinates of a configuration, so in particular $\homeo$ and $\homeo[\R]$ do not act transitively. As a matter of convention, we define $\pconf[S^1]{n}$ and $\pconf[\R]{n}$ to be the connected components consisting of configurations with a fixed cyclic or linear ordering. 

\begin{eg}\label{eg:confcompute}
    For any topological space, $\conf[X]{1}\cong\pconf[X]{1}\cong X$. Now, we consider $\pconf[S^1]{2}$ and $\conf[S^1]{2}$. In particular
    $$\pconf[S^1]{2} = \{(\theta, \varphi)\in S^1\times S^1 = T^2\,\vert\, \theta\neq\varphi\}$$ Considering $T^2$ as the unit square with edge identifications, $\pconf[S^1]{1}$ is then the complement of the line $\theta=\varphi$, which is easily seen to be homeomorphic to an open annulus. In this model, the symmetric group $S_2$ acts by a reflection over the diagonal. A simple way to identify the homeomorphism type of the quotient by this action is to note that it is a noncompact surface fibering over the circle 2-fold covered by an annulus; thus, it is a M\"{o}bius band\cite{morton:SymmetricProducts}.
\end{eg}

A foundational result about actions of $\homeo[M]$ from \cite{chen:StructureTheorems} is that orbits are continuous injective (equivariant) images of admissible covers of $\conf[M]{n}$. More precisely, they show for any $x\in N$ that there is a finite point set $X\subset M$ such that $\stab{X}\supseteq G_x \supseteq \pstab{X}$ and thus, since an orbit of a group action is a continuous injective image of the quotient of the group by the stabilizer of a point in the orbit, the orbit topologies sit between $\homeo[M]/\stab{X}$ and $\homeo[M]/\pstab{X}$. They prove that $\homeo[M]/\stab{X}\cong \conf[M]{n}$ and that $C_X\coloneqq \homeo[M]/\pstab{X}$ covers $\conf[M]{n}$ with deck group $\stab{X}/\pstab{X}$. Thus any orbit of $\homeo[M]$ is the continuous injective image of an intermediate covering between $\homeo[M]/\pstab{X}\to \conf[M]{n}$. The action of $\homeo[M]$ is given by left multiplication. See Section 4 of Chen-Mann\cite{chen:StructureTheorems} for further details. Otherwise, an equivariant description of all orbits of a $\homeo$ action on a closed three-manifold will suffice for our purposes.

\begin{figure}
    \begin{minipage}{.49\textwidth}
        \centering
        \includesvg[width=0.4\linewidth]{figures/3torusdiagonalaction}
        \caption{Orbit decomposition for the diagonal action on $T^3$}
        \label{fig:3torusdiagonalaction}
    \end{minipage}\hfill
    \begin{minipage}{.49\textwidth}
        \centering
        \includesvg[inkscapelatex=false, width=0.4\linewidth]{figures/3dorbitidentifications}
        \caption{Identifications marked on the glued faces of one of the two 3-dimensional orbits}
        \label{fig:3dorbitidentifications}
    \end{minipage}
\end{figure}

\begin{prop}\label{prop:orbit-computation}
    Suppose that $\homeo$ acts on $M$ closed 3-manifold, then every orbit a global fixed point or an equivariant continuous injective image of:
    \begin{enumerate}
        \item $\conf[S^1]{1} \cong \pconf[S^1]{1}\cong S^1$
        \item $\pconf[S^1]{2} \cong (0,1)\times S^1$, the open annulus
        \item $\conf[S^1]{2} \cong M\ddot{o}b$, the open M\"{o}bius band,
        \item $\pconf[S^1]{3} \cong D^2\times S^1$, the 3-dimensional open solid torus
        \item $\conf[S^1]{3} \cong D^2\times S^1$ --- restriction to $\SO$ gives the $(3, 1)$ standard fibered solid torus
    \end{enumerate}
\end{prop}
\begin{proof}[Proof of \cref{prop:orbit-computation}]
    The overall structure of the proof is as follows. First, we reduce to a finite list of possible orbit topologies. Then the first three bullet points follow from \cref{eg:confcompute}. Finally, the easiest way to observe the last two bullet points is to study the three-dimensional orbits of the diagonal action of $\homeo$ on $T^3$ in coordinates, and then consider a quotient of $T^3$ which is compatible with this action.  

    Since $\pi_1(\homeo) \cong \Z \cong \pi_1(\pconf[S^1]{n})$, every admissible cover of $\conf[S^1]{M}$ is an intermediate cover of the covering $\pconf[S^1]{n}\to \conf[S^1]{n}$. By invariance of domain, the $\homeo$ orbits are at most 3-dimensional (and $\conf[S^1]{n}$ is $n$-dimensional), so $n=1,\; 2,\; 3$. The $n=1,\; 2$ cases are addressed in \cref{eg:confcompute}. So it remains only to study the $n=3$ case. 

    Consider the diagonal action of $\homeo$ on $T^3=[0,1]^3/\sim$, i.e. $$\rho(f)(\theta, \varphi, \psi) = (f(\theta), f(\varphi), f(\psi))$$ This action has one 1-dimensional orbit $\sigma \coloneqq \{(\theta,\theta,\theta)\,\vert\,\theta\in S^1\}$ and three 2-dimensional orbits corresponding to each torus $\theta=\varphi$, $\varphi=\psi$ and $\theta=\psi$ minus $\sigma$. To determine the number and nature of three-dimensional orbits, it is necessary to study the complement of these lower-dimensional orbits in $T^3$. First, the complement of the one and two dimensional orbits intersected with $(0,1)^3$ is a disjoint union of six tetrahedra (see \cref{fig:3torusdiagonalaction}). To return to $T^3$, the identifications of the faces of $[0,1]^3$ provide gluing instructions for the faces of the tetrahedra. This collection of tetrahedra glue together in two cycles to give two disjoint subsets of $T^3$ each homeomorphic to $D^2 \times S^1$ corresponding to the two possible cyclic ordering of three marked points on $S^1$ (see \cref{fig:3dorbitidentifications} for the identifications for one component). By restricting to one of these components we recover the configuration space action of $\homeo$ on $\pconf[S^1]{3}\cong D^2 \times S^1$. The rotation subgroup action restricts directly from the rotation subgroup action on $T^3$, so in particular, all fibers of the associated Seifert-fibered structure are regular. The action of $S_3$ which permutes coordinates can be generated by a $2\pi/3$ rotation about the line $x=y=z$. In particular, the quotient by this action identifies each of the three tetrahedra which make up $\pconf[S^1]{3}$. Thus, $\conf[S^1]{3}$ can be understood as a tetrahedron with all but two faces (including edges and vertices) cut out and the two remaining faces glued by a 1/3 twist. The rotation subgroup action rotates along circular fibers parallel to the edge opposite the shared edge of the two glued face. In particular, this description coincides with the construction of the $(3, 1)$ standard fibered solid torus from the theory of Seifert fibrations. The covering from $\pconf[S^1]{3}$ to $\conf[S^1]{3}$ is three-fold, so there is no intermediate covering.
\end{proof}


% {\bf Product actions:} Let $\rho:\homeo[S^1]\to\homeo[\Sigma]$ where $\Sigma$ is some compact surface. Then there are two associated essentially different actions of $\homeo[S^1]$ on $\Sigma\times S^1$:
%     \begin{alignat*}{2}
%         \rho\times 0(f)(p, \theta) &\coloneqq (\rho(f)(p), \theta) \\
%         \rho\times \text{ev}(f)(p, \theta) &\coloneqq (\rho(f)(p), f(\theta))
%     \end{alignat*}
%     The first action is not particularly interesting and is mentioned only for completeness. If $\rho$ has fixed points, then $\rho\times 0$ does as well, while $\rho\times \text{ev}$ is fixed-point free. The orbits of this action can be understood as follows. Given a global fixed point for $p$ for the action $\rho$, $\{p\}\times S^1$ is a 1-dimensional orbit. Given $\sigma$ a 1-dimensional orbit, $\sigma\times S^1$ is equivariantly homeomorphic to a 2-torus with the diagonal action. Finally the closure of a two dimensional orbit $\bar{X}\times S^1$ is equivariantly homeomorphic to $T^3$ with the diagonal action split along an invariant $T^2$ (or the corresponding 3-fold quotient if $X$ is a m\"{o}bius band) 


\section{General \texorpdfstring{$\homeo[M]$}{Homeo\_0(M)} action structure: weak tube theorem}\label{sec:orbit-bundle-lemma}
Though the primary focus of this paper concerns actions on 3-manifolds, in an effort to understand how orbits of a fixed dimension are situated in the ambient manifold we arrive at a generalized flat bundle structure. This result can also be thought of as a direct generalization of Theorem 1.5 in \cite{chen:StructureTheorems}, which says that when $\dimn{N} < 2*\dimn{M}$ then $N$ is a generalized flat bundle over $M$ foliated by the orbits of $\homeo[M]$. The primary result of this section is as follows:
\orbitbundle*

We say this is a generalization since the hypothesis that the action has only orbits of a fixed dimension is a direct consequence of $\dimn{N} < 2*\dimn{M}$. The proof follows a very similar structure to that of Mann and Chen's Theorem 1.5 as well, most of the divergence from this structure come in the proof of the continuity and properness of the map from the model spaces to $N$. \cref{subsec:orbit-bundle-construction} gives the necessary constructions to state \ref{thm:orbit-bundle} more precisely. \cref{subsec:top-lemmas} provides a key topological lemma (and a closely related one) which is used in the proof of \cref{thm:orbit-bundle} in \cref{subsec:orbit-bundle-proof}. Finally \cref{subsec:example-orbit-bundle-classif} provides an application of \cref{thm:orbit-bundle}, classifying up to equivariant homeomorphism all actions of $\homeo$ on 3-manifolds having only 2-dimensional orbits.

\subsection{\texorpdfstring{$\homeo[M]$}{Homeo\_0(M)} invariant tube construction}\label{subsec:orbit-bundle-construction}
Let $N$ be a connected manifold of dimension $D=n*\text{dim}(M)$ for some $n$. Let $\rho:\homeo[M]\to\homeo[N]$ be some action such that all orbits have dimension $n$. In particular, by the Chen-Mann orbit classification theorem, all of the orbits are injective continuous equivariant images of covers of $\conf[M]{n}$.
Fix a base point $b\in \conf[M]{n}$.
We frequently need to refer to both a point in $\conf[M]{n}$
and the associated set of points in $M$.
To avoid confusion,
if $x$ is a point in $\conf[M]{n}$,
we will denote by $x_\set$\improve{how about underline instead? this reads pretty clunky} the corresponding set of points in $M$.
Also,
note that the deck group of the maximal admissible covering $\maxcov \to \conf[M]{n}$ can be identified with $\stab{b_\set}/\pstab{b_\set}\subset\homeo[M]$.

To begin, we construct the flat bundle structure of interest. The first step is defining the map which will serve as the projection map of our bundle structure. Suppose $O$ is an orbit in $N$.
Recall that $O$ is a continuous injective image
of an admissible cover of $\conf[M]{n}$
with a lift of the standard configuration action.
So, in particular,
for every point $x\in O$,
there is a unique $X\in\conf[M]{n}$
such that $\pstab{X} \subseteq G_x \subseteq \stab{X}$.
Then, define $p(x)_\set = X$.
Since $G_{\rho(f)(x)} = f G_x f^{-1}$
it follows that $\stab{f(X)} \subseteq G_{\rho(f)(x)}\subseteq \stab{f(X)}$.

\begin{prop}
    The map $p: A\to \conf[M]{n}$ is $\homeo[M]$ equivariant and continuous.
\end{prop}
\begin{proof}
    Equivariance follows directly from the definition.
    So, the primary concern of this result is to show that $p$ is continuous.
    We will do this by checking continuity on a nice basis for the configuration space topology.
    In particular,
    note that if $\mathscr{B}$ is a basis for the topology on $M$,
    then the set of all products of $n$ disjoint elements of $\mathscr{B}$ form a basis for the topology on $\pconf[M]{n}$
    i.e., the $n$-fold product of $M$ with itself minus the fat diagonal.
    Then, the image of this basis under the quotient forms a basis for the topology on $\conf[M]{n}$.
    Let $U$ be the image of one such $B_1\times\dots\times B_n$
    under the quotient map to $\conf[M]{n}$.

    \begin{clm*}
        Let $H\cdot B_i$ denote the $\homeo[M]$ subgroup of homeomorphisms supported on $B_i$.
        $$p^{-1}(U) = N - \bigcup \text{fix}(H\cdot B_i)$$
    \end{clm*}
    \begin{proof}
        By the equivariance of $p$
        and the \textit{locally continuously transitive} property of homeomorphism group actions
        $$\text{fix}(H\cdot B_i) = \{x\in N\, |\, p(x)_\set\cap B_i = \emptyset\}$$
        Interpreted semantically,
        this says that
        the fixed point set of $H\cdot B_i$ acting on $N$ is
        the set of all points whose image under $p$ has no coordinate in $B_i$.
        Then we have the following chain of equivalences:
        \begin{align*}
            N - \bigcup_{i=1}^n\text{fix}(H\cdot B_i) &= N - \bigcup_{i=1}^n\{x\in N\, |\, p(x)_\set\cap B_i = \emptyset\}\\
            &= \bigcap_{i=1}^n N - \{x\in N\, |\, p(x)_\set\cap B_i = \emptyset\}\\
            &= \bigcap_{i=1}^n \{x\in N\, |\, p(x)_\set\cap B_i \neq \emptyset\}\\
            &= p^{-1}(U)
        \end{align*}
    \end{proof}
    So,
    $p^{-1}(U)$ is open
    since $\cup_{i=1}^n\text{fix}(H\cdot B_i)$ is a finite union of closed sets
    and thus closed.
    In particular,
    this shows that $p$ is continuous.
\end{proof}

Using the map $p$, we construct the following generalized flat bundle over $\conf[M]{n}$.

\begin{construction*}\label{const:orbit-bundle}
    Denote $F\coloneqq p^{-1}(b)$.
    Let $f\in\stab{b_\set}$, then under the configuration space action $f*b=b$, so by the equivariance of $p$, the action of $\stab{b_\set}$ preserves $F$.
    By Theorem 2.14 in \cite{chen:StructureTheorems}, the subgroup of $\homeo[M]$ which fixes $x\in p^{-1}(b)$ is contained in $\pstab{b_\set}$
    So in particular, $\pstab{b_\set}$ fixes $F$ pointwise.
    Thus, the action of $\stab{b_\set}$ on $F$ descends to an action of $\rho_F:\stab{b_\set}/\pstab{b_\set}\to \homeo[F]$. The group $\stab{b_\set}/\pstab{b_\set}$ is naturally isomorphic with $\Gamma$, the deck group of $C_{b_\set}\to \conf[M]{n}$.
    In particular, $\Gamma$ acts on $E_b \coloneqq\maxcov\times F$ by the deck group action on $\maxcov$ and $\rho$ on $F$.
    Then by definition, $E_b/\Gamma\to\conf[M]{n}$ is a generalized flat bundle.

    \begin{defn}
        Suppose $\rho:\homeo[M]\to\homeo[N]$ is an action as described at the beginning of the section. We refer to $E_b/\Gamma\to\conf[M]{n}$ as the generalized flat bundle associated to $\rho$
    \end{defn}

    Let $L$ denote the action of $\homeo[M]$ on $\maxcov$, then we have the product action $L\times 0$ on $E_b$. To show that this action descends to the quotient, it suffices to prove that the actions of $\homeo[M]$ and $\Gamma$ commute. The action $L\times 0$ is trivial on $F$, and deck transformations of $\maxcov\to \conf[M]{\lvert b_{\set}\rvert}$ commute with the left action of $\homeo[M]$ on $\maxcov$. So, in particular, the action of $\homeo[M]$ on $E_b$ descends to an action on $E_b/\Gamma$. When we refer to $E_b/\Gamma$, the implied action of $\homeo[M]$ will always be this action.
\end{construction*}

In light of this construction, the following is the precise statement of \cref{thm:orbit-bundle} which we will refer to as the weak tube theorem.

\begin{thm}\label{thm:precise-orbit-bundle}
    Suppose that $\homeo[M]$ acts on a manifold $N$ with orbits all having the same fixed dimension.
    Then, there's a map: $$I: \maxcov \times p^{-1}(b)/ \Gamma \to N.$$
    Moreover, $I$ is an equivariant homeomorphism
    with the $\homeo[M]$ action on $\maxcov \times p^{-1}(b) / \Gamma$ being the quotient of the product of the standard action $L$ on $\maxcov$
    with the trivial action on the fiber.
\end{thm}

\subsection{Three Useful Topological Lemmas}\label{subsec:top-lemmas}
The following are self-contained purely topological lemmas about homeomorphism groups and their actions. They are all used in the proof of the weak tube theorem \cref{thm:orbit-bundle}
% \begin{fact}\label{compact-converg-imply-cont}\improve{this can be omitted, it's ``standard''}
%     Suppose that $X, Y$ are metric spaces, $f_n: X\to Y$ is a sequence of continuous functions converging to $f$ uniformly on compact sets and $(x_n)$ is a sequence of points converging to $x$. Then $f_n(x_n)$ converges to $f(x)$
% \end{fact}
% \begin{proof}
%     A convergent sequence together with its limit is a compact set in any metric space, so $f_n$ converges uniformly on $A = \{x_n\}\cup\{x\}$. Following definitions, for any $\varepsilon > 0$, there exists some $N$ such that for all $n> N$ and $t \in A$, $d_Y(f_n(t), f(t)) < \varepsilon/2$. Similarly, by the convergence of $(x_n)\to x$ and the continuity of $f$, we can demand $N'$ such that for all $n>N'$, $d_Y(f(x_n), f(x)) < \varepsilon/2$. Then, by the triangle inequality, for all $n>\text{max}(N, N')$, $$d_Y(f_n(x_n), f(x)) \leq \varepsilon$$.
% \end{proof}

\begin{lem}\label{lem:union-over-cpct}
    Suppose there is some action $\rho$ of $\homeo[M]$ on $N$. Suppose $A\subset N$ is closed and $H\subset\homeo[M]$ is compact, then $H\cdot A \coloneqq \bigcup_{h\in H} \rho(h)(A)$ is closed. If $A$ is compact, then $H\cdot A$ is also compact.
\end{lem}

\begin{proof}
    Suppose $(x_n)\to x$ is a convergent sequence in $H\cdot A$, then by definition, there is a sequence of homeomorphisms $f_n\in \homeo[M]$ such that $\rho(f_n^{-1})(x_n)\in A$ for all $A$. Since $H$ is compact, we may pass to a convergent subsequence of $f_{k_n}\to f\in H$. Since convergence in $\homeo[M]$ is uniform convergence on compact sets and compact convergence implies continuous convergence, $\rho(f^{-1}_{k_n})(x_{k_n}) \to \rho(f^{-1})(x)$. Since $A$ is closed, $\rho(f^{-1})(x)\in A$, thus $x\in H\cdot A$.

    Suppose further that $A$ is compact. Let $\rho(f_n)(x'_n)=(x_n)\subset H\cdot A$ an arbitrary sequence. By sequential compactness of $A$, there is some subsequence such that $(x'_{n_k})\to x\in A$. Passing to a further subsequence, we may also assume that $f_{n_k}\to f\in H$. Then, by continuous convergence, $\rho(f_{n_k})(x_{n_k})$ converges to $\rho(f)(x)\in H\cdot A$. Thus, $H\cdot A$ is sequentially compact, and thus compact since $N$ is metrizable.
\end{proof}

The following is a closely related technical result which considers convergence of sets in the sense of Kuratowski\cite{kuratowski:TopologyI}. In case this is unfamiliar to the reader, we reproduce the relevant definitions below. 
\begin{defn}
    Suppose $(A_n)$ is any sequence of subsets of some topological space. We define the following two sets
    \begin{align*}
        \li{A_n}&\coloneqq\{x\in X\,\vert\, x_n\in X_n;\, (x_n)\to x\} \\
        \ls{A_n}&\coloneqq\{x\in X\,\vert\, x_{n_k}\in X_{n_k};\, (x_{n_k})\to x\}
    \end{align*}
    known as the lower and upper limits of the sequence respectively (see \cite{kuratowski:TopologyI} \S29.I and \S29.III). When these sets coincide, the sequence is said to be convergent (in the sense of Kuratowski) with limit $\klim{A_n} \coloneqq \li{A_n} = \ls{A_n}$ (see \cite{kuratowski:TopologyI} \S29.VI).
\end{defn}

This definition is rather convenient in this setting as it enables discussion about limits of sequences of orbits in a metric-free way. The following lemma will be used later on to prove \cref{lem:seq-of-annuli}, which in turn can be used to begin to characterize the topological boundary of $\homeo$ tubes.

\begin{lem}\label{lem:union-over-cpct-k-lim}
    Suppose there is some action $\rho$ of $\homeo[M]$ on $N$. Suppose $X_n \subset N$ is a convergent sequence of sets $\klim{X_n} = X$ and $H\subset\homeo[M]$ is compact, then $\klim{H\cdot X_n} = H\cdot X$.
\end{lem}
\begin{proof}
    By relation 0 in \cite{kuratowski:TopologyI} \S29.V, $\li{A_n}\subseteq\ls{A_n}$; so, it will suffice to show $\ls{(H\cdot X_n)}\subseteq H\cdot \klim{X_n}\subseteq\li{(H\cdot X_n)}$. First, we check $\mathit{\ls{X_n}\subseteq H\cdot\klim{X_n}}$. Suppose that $x\in \ls{H\cdot X_n}$. That is, there is some $n_1 < \dots < n_k < \dots$ such that there are $(x_{n_k})$ chosen from $X_{n_k}$ and $f_{n_k}\in\homeo[M]$ such that $\rho(f_{n_k})(x_{n_k})$ converges --- call the limit point $x$. Since $H$ is compact, up to passing to some further subsequence, $f_{n_k}$ converges to some $f\in\homeo[M]$. Thus, $\rho(f)(x_{n_k})$ converges to $x$. In particular, $x_{n_k}$ converges to some $x'$ which is by definition in $\ls{X_n} =\klim{X_n}$, so $x=\rho(f)(x')$ lies in $H\cdot \klim{X_n}$. To check the remaining containment, suppose that $x\in H\cdot\klim{X_n}$. In particular, there is some sequence $(x_n)$ chosen from each $X_n$, some $f\in\homeo[M]$ and some $x'$ such that $(x_n)$ converges to $x'$ and $\rho(f)(x')=x$. Then, by continuity, $\rho(f)(x_n)\to\rho(f)(x')=x$, so by definition, $x\in\li{X_n}$. So, in conclusion, $H\cdot X_n$ is convergent and $\klim{(H\cdot X_n)} = H\cdot\klim{X_n}$.
\end{proof}

To make use of these lemmas in a general setting, we have the following construction of a particular compact family of homeomorphisms which is used in the proof of \cref{thm:precise-orbit-bundle}.

\begin{lem}\label{lem:cpct-to-basepoint-confM}
    Suppose $K\subseteq\conf[n]{M}$ is compact and $b\in\conf[n]{M}$ is a choice of basepoint, then there is a compact set $H_K\subset\homeo[M]$ such that for every $p\in K$, there is some $f\in H_K$ such that $f*p=b$ where $*$ denotes the configuration space action.
\end{lem}

The details are not particularly enlightening but are presented for completeness. The proof follows a local-global structure, first proving for $\R^n$, generalizing to $M$ and finally to $\conf[n]{M}$.

\begin{lem}\fix{katie says this lemma can be simplified}
    Let $B_r(0)$ denote the radius $r$ ball in $\R^n$ centered on 0. There exists a compact set $H\subset\homeo[R^n]$ such that for every $p\in \cl{B_1(0)}$ there is some $f\in H$ such that $f(p)=0$ and $\text{Supp}(H) \subseteq B_2(0)$
\end{lem}
\begin{proof}
    For the following we use generalized spherical coordinates $(r, p)$ on $\R^n$ where $p\in S^{n-1}$. Denote the dilation map with ratio $r$ and center $p$ as $D_p^r$ and the antipodal map of $S^{n-1}$ as $\alpha$. Associated to each $(\rho, \sigma)\in \cl{B_1(0)}$ define the homeomorphism $f_{(\rho, \sigma)}\in\homeo[\R^n]$ as follows:
    \[f_{(\rho,\sigma)} \coloneqq
    \begin{cases}
        \text{Id} & r>2\\[.8em]
        D_{(r,\alpha(\sigma))}^{\frac{1-\rho}{2}r+\rho}(r,p) & 2\geq r \geq 1 \\[1em]
        D_{(1, \alpha(\sigma))}^{\frac{1+\rho}{2}}(r,p) & r < 1
    \end{cases}
    \]
    Clearly $f_{(\rho, \sigma)}((\rho,\sigma))=0$; moreover, the map $f_{(\rho,\sigma)}$ depends continuously on $(\rho,\sigma)$, so $H=\{f_{(\rho,\sigma)}\,\vert\, (\rho,\sigma)\in\cl{B_1(0)}\}$ is compact.
    Note that many other choices of maps are possible.
\end{proof}

\begin{lem}\label{lem:cpct-to-basepoint-M}
    Suppose $M$ is an $n$-manifold, $K\subseteq M$ is compact, and $b\in M$ is a basepoint, then there is a compact subset $H_K\subset\homeo[M]$ such that for every $p\in K$, there is some $f\in H_K$ such that $f(p)=b$.
\end{lem}
\begin{proof}
    Let $\{(U_\alpha, \varphi_\alpha)\}$ be a regular atlas where $U'_\alpha$ denotes the enlarged open neighborhood of $U_\alpha$, and $\varphi_\alpha:U'_\alpha \to B_3(0)\subseteq \R^n$, and $\varphi_\alpha:\cl{U_\alpha} \to \cl{B_1(0)}$. Restrict to a finite subcover $\{(U_i, \varphi_i)\}$ of $K$. Let $H_i$ denote the pullback of the collection $H$ by $\varphi_i$, extended by the identity on $M-\varphi_i^{-1}(B_2(0))$. Clearly $\varphi_i$ induces a homeomorphism of the subgroup of $\homeo[M]$ with support on $U_i'$ and the subgroup of $\homeo[\R^n]$ with support on $B_3(0)$, so $H_i$ is compact. Finally, let $f_i\in\homeo[M]$ be such that $f_i(\varphi_i^{-1}(0)) = b$, then $$H_K = \bigcup_{i=1}^n H_i\cdot f_i.$$
    This is a finite union of compact sets, so it is compact, moreover there are clearly element which take each point in $K$ to the basepoint.
\end{proof}

Finally, we prove \cref{lem:cpct-to-basepoint-confM} using these intermediate results.

\begin{proof}
    Suppose $d$ is a metric on $M$, then clearly there is a uniform $\varepsilon$ such that for all $p\in K$, and $x, y\in p_\set$, $d(x,y)>\varepsilon$. In particular, by choosing an atlas for $M$ with all charts having diameter bounded above by $\varepsilon/2$, we can cover $K$ with finitely many charts $V_j \coloneqq [U_{1,\,j}\times\dots\times U_{n,\, j}]$ where the $U_i$'s are mutually disjoint charts for $M$. Then let $H^j_i$ be the compact family given by \cref{lem:cpct-to-basepoint-M}. Since all the $U_i$'s are mutually disjoint, $H_j\coloneqq H_{1,\,j}\cdot\ldots\cdot H_{n,\, j}$ is a compact family of homeomorphisms taking all points in $\cl{V_j}$ to $p_j\coloneqq\{\varphi_{1,\, j}^{-1}(0),\dots, \varphi_{n,\, j}^{-1}(0)\}$. Since the action of $\homeo[M]$ on $M$ is $n$-transitive for all $n$, there is an element $f_j$ of $\homeo[M]$ such that $f_j*p_j = b$. Then, the union of $H_j\cdot f_j$ is a compact family of homeomorphisms having the desired property for $K\subseteq M$.
\end{proof}

\subsection{Proof of weak tube theorem}\label{subsec:orbit-bundle-proof}
The remainder of the proof is concerned with proving \cref{thm:precise-orbit-bundle}, namely that when $N$ is a manifold with an action of $\homeo[M]$ with orbits of fixed dimension, then $N$ is equivariantly homeomorphic to $\maxcov\times F/\Gamma$, where the action on the quotient is as described in the construction preceding \cref{thm:precise-orbit-bundle}. The following proof structure closely follows Chen and Mann's \cite{chen:StructureTheorems} proof of their Theorem 1.5. The proof substantially diverges in the verification that the constructed map is in fact a homeomorphism.

\begin{proof}
    In order,
    we will construct a map $\hat{I}$ from $\maxcov \times F$ to $N$,
    show that $\hat{I}$ is well-defined,
    descends to an equivariant map $I$ on the quotient,
    $I$ is continuous, and
    $I$ is proper (thus a homeomorphism).
    Fix a lift $\tilde{b}$ of $b$ in $\maxcov$.

    \medskip
    {\it Definition of $\hat{I}$.} Let $(x, y)\in \maxcov\times F$, then
    suppose $f\in\homeo[M]$ such that $L(f)(x) = \tilde{b}$,
    then define $\hat{I}((x,y)) = \rho(f^{-1})(y)$. Suppose that $g$ is another element of $\homeo[M]$ such that $L(g)(x) = \tilde{b}$. Then, $gf^{-1}$ fixes $\tilde{b}$. Since $L$ commutes with the action of $\Gamma$ on $\maxcov$ by deck transformations, that $gf^{-1}$ fixes a particular lift of the base point suffices to show $gf^{-1}$ is in the pointwise stabilizer of $b_{\set}$. In particular, under the action on $N$, it fixes $p^{-1}(b)$ pointwise, so:
$$\rho(f^{-1})(y) = \rho(f^{-1})(\rho((gf^{-1})^{-1})(y)) = \rho(g^{-1})(y)$$

    \medskip
    {\it $\hat{I}$ descends.} Suppose that $\gamma\in\Gamma$ is a deck transformation of $\maxcov\to \conf[M]{n}$. Recall that $\Gamma$ is isomorphic to $\stab{b_\set}/\pstab{b_\set}$, so let $\bar{\gamma}\in\stab{b_\set}\subseteq\homeo[M]$ be an element which represents the $\pstab{b_\set}$ equivalence class of the element corresponding to $\gamma$. Note that the action of $\homeo[M]$ on $\maxcov$ commutes with deck transformations since it is a lift; in particular, for any $f\in\homeo[M]$ and any $x\in\maxcov$:
    $$L(f\circ\bar{\gamma})(x) = L(f)(\gamma(x)) = \gamma(L(f)(x)) = L((\bar{\gamma}\circ f)(x).$$
    Then, suppose that $f\in\homeo[M]$ satisfies $L(f)(x) = \tilde{b}$. By the above equalities, $L(f\bar{\gamma}^{-1})(\gamma(x)) = L(f)(x) = \tilde{b}$, so $$\hat{I}(\gamma(x, y)) = \rho((f\bar{\gamma}^{-1})^{-1})(\rho(\bar{\gamma})(y)) = \rho((\bar{\gamma}^{-1}f)^{-1})(\rho(\bar{\gamma})(y))$$
    Finally, by rearranging the elements acting on $y$, we arrive at $\hat{I}(\gamma(x,y)) = \rho{f}(y) = \hat{I}(x,y)$. Thus, $\hat{I}$ descends to the quotient -- call the resulting map $I$.

    \medskip
    {\it $I$ is equivariant.}
    Let $\varphi\in\homeo[M]$, $[(x,y)]\in E_b/\Gamma$ and $f_x$ a defining map for $\hat{I}((x,y))$. Note that $f_x\varphi^{-1}$ is a defining map for $\hat{I}(L(\varphi)(x), y)$, so,
    \begin{align*}
        I(\varphi*[(x,y)]) &= I([\varphi*(x, y)]) \\
                        &= I([(L(\varphi)(x), y)]) \\
                        &= \hat{I}((L(\varphi)(x), y)) \\
                        &= \rho(\varphi f_x^{-1})(y) \\
                        &= \rho(\varphi)(\hat{I}((x,y))) \\
                        &= \rho(\varphi)(I((x, y)))
    \end{align*}
    Thus, $I$ is equivariant.

    \medskip
    {\it $I$ is bijective.} Note that since $I$ is equivariant, the image is clearly a $\homeo[M]$ invariant set, so to see that $I$ is surjective, it suffices to note that the image of ${\tilde{b}}\times F$ is exactly $F\subseteq N$, which intersects every orbit at least once, so the image of $I$ is $N$.

    Now, to show injectivity, suppose $(x,t)$, $(y, s)$ are points in $\maxcov\times F$ such that $\hat{I}(x,t) = \hat{I}(y, s)$. In particular, there exist $f_x, f_y\in\homeo[M]$ such that $L(f_x)(x) = \tilde{b}$, $L(f_y)(y) = \tilde{b}$ and $\rho(f_x^{-1})(t) = \rho(f_y^{-1})(s)$. Then, $\rho(f_y f_x^{-1})(t)= s$, so in particular, $f_y f_x^{-1}\in\stab{b_{\set}}$. Let $\gamma$ denote the deck transformation of $\maxcov\to\conf[M]{n}$ associated to $f_y f_x^{-1}$. Then, $L(f_y f_x^{-1})(\tilde{b}) = \gamma^{-1}(\tilde{b})$. By left cancellation and commuting $\gamma$ through $L$, $\gamma(L(f_x^{-1})(\tilde{b})) = L(f_y^{-1})(\tilde{b})$. So, by definition of $f_x$ and $f_y$, $\gamma(x) = y$. Thus, under the deck group action, $f_y f_x^{-1}$ takes $(x, t)$ to $(y, s)$, so they represent the same point in the quotient, so $I$ is injective.

    \medskip
    {\it $I$ is continuous.} We will now check sequential continuity. Suppose $[(x_i, y_i)]$ converges to $[(x, y)]$ in $(\maxcov\times F)/\Gamma$ where $(x_i, y_i)$ is a sequence of lifts which converges to $(x,y)$ in $E_b$. Let $(x_i^1,\dots x_i^n)\in\conf[M]{n}$ be the projections of the $x_i$'s such that $x_i^j$ converges to $x^j$ where $(x^1,\dots, x^n)$ is the $\conf[M]{n}$ projection of $x$. there exists some ball $B\ni x$ and some $N$ such that for all $n> N$, $x_n\in B$. The $\conf[M]{n}$ and subsequently $M$ projection of $B$ is (up to shrinking) a disjoint union of n disjoint balls $B_N^1,\dots,B_N^n$ in $M$ such that $B_N^j$ contains $x_i^j$ and $x^j$ for all $i>N$. Finally, let $B_N^j\supset\dots B_{N+1}^j\supset\dots$ be a decreasing sequence of balls with diameter (in some generic fixed metric on $M$) converging to zero defined analogously to $B_N^j$. Then for all $i\geq N$ let $\varphi_i^j\in\homeo[M]$ be such that $\varphi_i^j(x_i^j)=x^j$ and $\text{supp}(\varphi_i^j)\subseteq B_i^j$ as guaranteed by the {\it locally continuously transitive} property described in \cref{prop:lct}. Define $\varphi_i \coloneqq \varphi^1_i\circ\dots\varphi^n_i$. Then if $f_x$ is a defining map for $\hat{I}((x, y))$ then $f_{x_i}\coloneqq f_x\circ\varphi_i$ is a defining map for $\hat{I}(x_i, y_i)$. Clearly $L(\varphi_i)(x_i) = x$ and $\varphi_i$ converges to $\text{Id}\in\homeo[M]$, so $f_{x_i}$ converges to $f_x$. Then, by continuity of the action, $\rho(f_{x_i}^{-1})(y_i)$ converges to $\rho(f_x^{-1})(y)$
    % Then, we are free to choose a sequence $\{f_{x_i}\} \subset \homeo[M]$ converging to $f_x$ a representative function for producing $I(x, y)$. Then, by the continuity of the $\homeo[M]$ action $\{\rho(f_{x_i})\}$ is a sequence converging to $\rho(f_x)$ in $\homeo[N]$. In particular, since $N$ is a manifold, convergence in $\homeo[N]$ implies pointwise convergence, so $$I(x_i, y_i) = \rho(f_{x_i}^{-1})(y_i) \to \rho(f_{x}^{-1})(y) = I(x, y).$$

    \medskip
    {\it $I$ is a homeomorphism.} A useful general topological fact is that a proper continuous map into a locally compact Hausdorff space is closed, so in particular a bijective, continuous, proper map into a locally compact Hausdorff space is a homeomorphism. So, to complete this proof, we will prove that $I$ is proper. The definition of properness we will use is that the image of a sequence which {\it diverges to infinity} must {\it diverge to infinity}.
    \begin{defn}
        A sequence $(x_n)$ in topological space $X$ is said to diverge to infinity if any compact subset $K\subset X$ contains at most finitely many points of $(x_n)$.
    \end{defn}
    Suppose $(x_n)$ is a sequence in $(\maxcov\times F)/\Gamma$ diverging to infinity, then let $K\subset N$ be a compact set. If $p(K)$ contains finitely many points of $\{p(I(x_n))\}$, then we are done, so suppose that $p(K)$ contains infinitely many points of $\{p(I(x_n))\}$. Pass to the subsequence of points whose image under $p$ is contained in $p(K)$ and reindex it as $(x_n)$. By \cref{lem:cpct-to-basepoint-confM}, there is a compact family of homeomorphisms $H\subseteq \homeo[M]$ such that for every point in $p(K)$, there is a corresponding element which takes that point to $b\in\conf[M]{n}$ under the configuration space action. By \cref{lem:union-over-cpct}, $H\cdot K\coloneqq \bigcup_{h\in H_K}\rho(h)(K)$ is a compact set containing $K$. In particular, $H\cdot K\cap p^{-1}(b)$ is a compact set.

    \begin{lem*}
        The map $I$ restricted to the image of $\tilde{b}\times F$ under the quotient (which we will call $\hat{F}$) is a homeomorphism onto $F\subset N$.
    \end{lem*}
    \begin{proof}
        Trivially $\hat{I}$ is a homeomorphism from $\tilde{b}\times F$ to $F$.
        No points in $\tilde{b}\times F$ are identified under the quotient, so the quotient map restricted to $\tilde{b}\times F$ is a homeomorphism.
        But, $I$ restricted to the image of $\tilde{b}\times F$ under the quotient is equal to
        $\hat{I}$ composed with the inverse of the restricted quotient map, which is a composition of homeomorphisms, so in particular
        $I\mid_{\hat{F}}: \hat{F} \to F$ is a homeomorphism.
    \end{proof}

    \begin{note*}
        Suppose $\homeo[M]$ acts on manifold $N$ and $(x_n)\subset N$ diverges to infinity, $H\subseteq\homeo[M]$ compact and $\{f_n\}\subset H$, then $(f_n*x_n)$ diverges to infinity. This follows directly from \cref{lem:union-over-cpct} since if $K$ is a compact set containing infinitely many points of $(f_n*x_n)$, then $H^{-1}(K)$ is a compact set containing infinitely many points of $(x_n)$.
    \end{note*}

    Let $f_n\in H$ be homeomorphisms such that $f_n*x_n\in \hat{F}$. By the preceding lemma, $(I(f_n*x_n))$ diverges to infinity. By equivariance, $I(x_n) = \rho(f_n^{-1})(I(f_n*x_n))$, so by the preceding note $I(x_n)$ diverges to infinity.
\end{proof}

\subsection{Application of \texorpdfstring{\cref{thm:orbit-bundle}}{Weak Tube Theorem}: Classification for \texorpdfstring{$M=S^1$}{M=S\string^1}}\label{subsec:example-orbit-bundle-classif}
For components consisting of codimension 1 orbits,
the fiber must be 1-dimensional
and all 1-dimensional homology manifolds are manifolds.
So,
equivariant homeomorphism types of codimension 1 $\homeo[M]$ invariant tubes are classified
by actions of the deck group of $\Gamma$
of the covering $C_X \to \conf[M]{n}$ on 1-manifolds
up to equivariant homeomorphism.
Note that,
in general, $\lvert \pi_0(F) \rvert \leq \lvert \Gamma \rvert$
and the components of $F$ are homeomorphic.

Since classification depends strongly on dimension constraints, a general result is not within reach,
but fixing $M = S^1$
and $\text{dim}(N)=3$ a classification is possible.
Note, in this setting $C_X = \pconf{2}$ and $\Gamma = \Z_2$. This proof will follow the convention of $\Z_2 \coloneqq \{-1, 1\}$ under multiplication. The following \cref{prop:orbit-bundle-computation} gives a complete classification of three-dimensional tubes about two-dimensional orbits of $\homeo$.

\begin{figure}
    \centering
    \includesvg[width=.95\linewidth]{figures/orbit bundle bases}
    \caption{$\SO$ projection of the orbit decomposition of the exceptional tubes types, cone points marked in red}
    \label{fig:orbit-bundle-bases}
\end{figure}

\begin{prop}\label{prop:orbit-bundle-computation}
    Suppose $\homeo$ acts on a 3-manifold $M$ with or without boundary with only 2-dimensional orbits. Then $M$ is equivariantly homeomorphic to $\conf{2}\times C$, $\pconf{2}\times C$ with the configuration space action cross the trivial action or one of four exceptional bundle structures described in terms Seifert-fibered structure and the action on the fiber.
    \begin{enumerate}
        \item $M$ is Seifert isomorphic to $M\ddot{o}b\times S^1$, and $F\cong S^1$. Considering the standard twisted $I$-bundle structure on $M\ddot{o}b$, the orbits of $\homeo$ are the $I$ fibers cross $S^1$.
        \item $M$ is Seifert isomorphic to $(\mathring{\mathbb{D}}^2, (2, 1))$, and $F\cong \R$. Action on the fiber is by a reflection.
        \item $M$ is Seifert isomorphic to $(\mathring{\mathbb{D}}^2, (2, 1), (2,1))$, and $F\cong S^1$. Action on the fiber is by the antipodal map.
        \item $M$ is Seifert isomorphic to $(\mathring{\mathbb{D}}^2\cup I, (2,1))$ where $I$ is a interval in the boundary of the closed disk $\mathbb{D}^2$, and $F \cong \lbrack 0,1\rbrack$. Action on the fiber is by a reflection.
    \end{enumerate}
    % The classification of $\homeo$ codimension 1 orbit bundles $X$ in 3-manifolds is as follows.
    % \begin{description}
    %     \item[$F$ disconnected] $F\cong C\sqcup C$ for a 1-manifold $C$, $X\cong \pconf{2}\times C$ and $\pi_{SO(2)}(X) \cong I \times C$
    %     \item[$F$ connected] \
    %     \begin{description}
    %         \item[$\Z_2$ acts on $F$ f.p. free] $F\cong S^1$, action is by a $\pi$ rotation. $X\cong \pconf{2} \tilde{\times} S^1$ and $\pi_{SO(2)} \cong M\ddot{o}b$\improve{add details to prove this}
    %         \item[$\Z_2$ acts on $F$ w/ fixed points] This depends on $F$, on each $\Z_2$ acts by reflection. Note in each case, the bundle is described in terms of the Seifert structure given by the $SO(2)$ action.
    %         \begin{description}
    %             \item[$F\cong \R$] $X$ is $(\mathring{\mathbb{D}}^2, (2, 1))$
    %             \item[$F\cong S^1$] $X$ is $(\mathring{\mathbb{D}}^2, (2, 1), (2,1))$
    %             \item[$F \cong \lbrack 0,1\rbrack$] $X$ is $(\mathring{\mathbb{D}}^2\cup I, (2,1))$ where $I$ is a interval in the boundary of the closed disk $\mathbb{D}^2$
    %         \end{description}
    %         \item[$\Z_2$ acts trivially] $X\cong\conf{2}\times C$ {\it N.B. resulting bundle is nonorientable (and top dimensional), so $N$ is as well}
    %     \end{description}
    % \end{description}
    % A generic component $C$ of the fiber is one of $S^1,\, \mathbb{R},\, [0,1]\, \text{or}\,  [0,1)$
\end{prop}
\setlength\extrarowheight{3pt}
\begin{table}[]
\begin{tabular}{rlll|ll}
\multicolumn{1}{l}{}       &                                              & \multicolumn{2}{c|}{product}                                                          & \multicolumn{2}{c}{exceptional}                                                                                 \\
\multicolumn{1}{l}{}       &                                              & \multicolumn{1}{l|}{orientable}                      & non-orientable                 & \multicolumn{1}{l|}{orientable}                                 & non-orientable                                \\ \cline{3-6}
                           & \multicolumn{1}{l|}{$F\cong S^1$}            & \multicolumn{1}{l|}{$\pconf{2}\times S^1$}           & $\conf{2}\times S^1$           & \multicolumn{1}{l|}{$(\mathring{\mathbb{D}}^2, (2, 1), (2,-1))$} & \multicolumn{1}{l|}{$M\ddot{o}b\times S^1$}   \\ \cline{2-6}
\multirow{-2}{*}{open}     & \multicolumn{1}{l|}{$F\cong \R$}             & \multicolumn{1}{l|}{$\pconf{2}\times \R$}            & $\conf{2}\times \R$            & \multicolumn{1}{l|}{$(\mathring{\mathbb{D}}^2, (2, 1))$}        & \multicolumn{1}{l|}{\cellcolor[HTML]{EFEFEF}} \\ \hline
                           & \multicolumn{1}{l|}{$F\cong \lbrack 0, 1 )$} & \multicolumn{1}{l|}{$\pconf{2}\times \lbrack 0, 1)$} & $\conf{2}\times \lbrack 0, 1)$ & \multicolumn{1}{l|}{\cellcolor[HTML]{EFEFEF}}                   & \multicolumn{1}{l|}{\cellcolor[HTML]{EFEFEF}} \\ \cline{2-6}
\multirow{-2}{*}{non-open} & \multicolumn{1}{l|}{$F\cong \cl{I}$}         & \multicolumn{1}{l|}{$\pconf{2}\times \cl{I}$}        & $\conf{2}\times \cl{I}$        & \multicolumn{1}{l|}{$(\mathring{\mathbb{D}}^2\cup I, (2,1))$}   & \multicolumn{1}{l|}{\cellcolor[HTML]{EFEFEF}} \\ \cline{2-6}
\end{tabular}
\caption{Summary of equivariant homeomorphism types of 3D tubes about 2D $\homeo$ invariant tubes.}
\label{table:orbit-bundles}
\end{table}



\begin{proof}
    By the weak tube theorem (\cref{thm:precise-orbit-bundle}), $M$ is equivariantly homeomorphic to $\pconf{2}\times F/\Z_2$ for some (not necessarily connected) one-manifold $F$ and some action of $\Z_2$ on $F$. The overall case structure of the result will come from first restricting to a finite list of possible fibers $F$ by bounding the number of components and then listing all of the possible $\Z_2$ actions on each example. Then, associated to each pair of fiber $F$ and $\Z_2$ action on $F$, there is a unique (up to $\homeo$-equivariant homeomorphism) 3-manifold $\pconf{2}\times F/\Z_2$, which is described explicitly.

    First, since the covering from $\pconf{2}\times F\to M$ is two-fold (as a quotient by a $\Z_2$ action) and $M$ is connected, $F$ has at most two connected components.
    Assume $F$ has two components.
    Since the quotient is connected,
    the action must permute the components (homeomorphically),
    so in particular we can express the fiber
    as $F\cong C\sqcup C$ for any one-manifold $C$.
    If $F$ has a single component,
    then the $\Z_2$ action is simply a choice of involution on $F$. Of course, any one-manifold $F$ admits a trivial action of $\Z_2$. There are exactly four non-trivial $\Z_2$ actions on one-manifolds, which can be listed as follows:
    multiplication by $-1$ on $[-1,1]$ and $\mathbb{R}$
    and either a half-turn rotation or the antipodal map on $S^1$. Note that the half-open interval admits no nontrivial involutions. The remainder of the proof will examine the three-manifold resulting from each fiber, $\Z_2$-action pair in the following order: $F=C\sqcup C$, connected fiber with trivial action, $S^1$ fiber with action by a half-turn rotation, $S^1$ fiber with action by the antipodal map, and finally $[-1,1]$ and $\mathbb{R}$ with the action by multiplication $-1$.


    In the case of disconnected fiber, the product $\pconf{2}\times F$ is disconnected, so the $\Z_2$ action on $\pconf{2}\times F$ is just a homeomorphism
    from one component to the other. In particular, the resulting quotent $\pconf{2}\times F/\Z_2$ is equivariantly homeomorphic $\pconf{2}\times C$
    where the resulting $\homeo$ action is the product of the standard configuration space action on $\pconf{2}$ and the trivial action on the fiber $C$.

    In the case of connected fiber $F$ with trivial $\Z_2$ action, since the action of $\Z_2$ on $\pconf{2}\times F$ is trivial on the $F$ factor,
    the map which sends $([x], p)$ to $[(x, p)]$ is a homeomorphism between $(\pconf{2}/\Z_2)\times F$ and $\pconf{2}\times F / \Z_2$,
    where $\Z_2$ acts as the deck group of $\pconf{2}\to\conf{2}$.
    Thus, the quotient $\pconf{2}\times F / \Z_2$ is ($\homeo$-equivariantly) homeomorphic to $\conf{2}\times F$.
    This is nonorientable (regardless of $F$) and top-dimensional,
    so in general,
    such tubes only arise when $N$ is nonorientable.
    
    These first two cases account for the `product' half of \cref{table:orbit-bundles}. The remaining four fiber and $\Z_2$ action pairs account for the exceptional actions. Consider as a first exceptional case $\Z_2$ acting on $S^1$ by a $\pi$ rotation. Then, $-1\in\Z_2$ acts by $((\theta, \varphi), \psi) \mapsto ((\varphi, \theta), \psi + \pi)$. This map is orientation-reversing, so the resulting quotient is non-orientable. The quotient fibers over the circle with annulus fibers, which are the $\homeo$ orbits. There is a unique twisted annulus bundle over the circle, and it is Seifert isomorphic to $M\ddot{o}b \times S^1$.

    As the next exceptional case, consider $\Z_2$ act on $S^1$ by the antipodal map. Then, $-1\in\Z_2$ acts by $((\theta, \varphi), \psi)\mapsto ((\varphi, \theta), -\psi)$. There are two $\SO$ preserved by this action, namely $\{(\theta, \theta + \pi)\}\times \{0, \pi\}$ (recall that the $\SO$ orbits are the circles $\theta-\varphi = c$ for some constant). Clearly all other fibers must be regular fibers, and these will be multiplicity 2 singular fibers. Clearly $\pconf{2}\times S^1$ has base surface Euler characteristic 0. So, the quotient base 2-d orbifold is an open 2-d orbifold with orbifold Euler characteristic 0, so the Euler characteristic of the underlying surface must be 1, in particular it is an open disk. Finally, by a cut-and-paste argument, we can show that the two singular fibers have $(2, 1)$ and $(2,-1)$-standard fibered solid torus neighborhoods respectively. So, in particular, the overall tube is Seifert isomorphic to $(\mathring{\mathbb{D}}^2, (2, 1), (2,-1))$

    Finally, consider $\Z_2$ acting on $\R$ or $[-1, 1]$ by a reflection. Then, $-1\in\Z_2$ acts by $((\theta, \varphi), x)\mapsto ((\varphi, \theta), -x)$. Similarly there is only one (multiplicity 2) singular fiber. The base surface of the product is a disk, thus has orbifold Euler characteristic 1, and the covering is degree 2, so the quotient base orbifold has orbifold Euler characteristic 1/2, so in particular by the same reasoning as the preceding computation, the underlying surface to the base orbifold is a disk. When the fiber is $\R$, this is an open disk and when the fiber is $[-1,1]$, the base surface for the product is an open disk together with two disjoint open intervals in the boundary. The quotient identifies these intervals, so the quotient disk has an open orbifold in the boundary. Since the same cut and paste construction shows the that the unique singular fiber has a $(2,1)$ standard fibered torus neighborhood, the overall tube is Seifert-isomorphic to $(\mathring{\mathbb{D}}^2, (2, 1))$.
\end{proof}

In fact, for $\text{dim}(N) = 4$, $n=3$, the classification is even simpler.

\begin{rem}
    The deck group is $\Z_3$, every action of $\Z_3$ on $\R$ is trivial, and every action of $\Z_3$ on $S^1$ is conjugate to the standard action by $2\pi/3$ rotations or trivial. Ultimately, after accounting for conjugacy, the only $\homeo$ invariant tubular neighborhoods about 3-dimenional orbits in a 4-manifold are $\conf[S^1]{3}\times \R$, $\conf[S^1]{3}\times S^1$ and $\pconf[S^1]{3}\times S^1$
\end{rem}

It is possible to continue this scheme to classify all $\homeo$ invariant tubular neighborhoods about codimension 1 orbits, but the details will be essentially number-theoretic in nature. The classification of $\homeo$ invariant tubes about codimension 2 orbit in 4-manifolds is tractable via the study of involutions on surfaces\cite{dugger:InvolutionsSurfaces}. A general codimension 2 classification depends on the significantly more involved theory of finite cyclic group actions on surfaces \cite{harvey:CyclicGroups}\cite{ding:ClassificationCyclic}.\fix{consider expanding}


\section{Structural results for \texorpdfstring{$\homeo$}{Homeo\_0(S\string^1)} acting on closed 3-manifolds}\label{sec:homeos1}

To understand $\homeo$ actions on closed 3-manifolds, it is instructive to at first have in mind which closed 3-manifolds may admit a $\homeo$ action.

\thmglobaltop*

Particularly in light of the weak tube theorem \cref{thm:orbit-bundle}, it is fruitful to divide the manifold $M$ being acted on into the sets of orbits of fixed dimension $M_{(1)}$, $M_{(2)}$ and $M_{(3)}$. The following $\homeo$ answer to the classical Lie group orbit type stratification theorem provides a nearly complete answer to the orbit gluing question when only one and two-dimensional orbits are present.

\weakotstratif*

To complete the story when $M_{(3)}=\emptyset$, we employ the following technical lemma:
\begin{lem}
    The frontier of every 2-dimensional orbit consists of one or two 1-dimensional orbit.
\end{lem}
It is interesting to note that the proof of this lemma uses very little which is particular to the dimension of the orbit and nothing that is particular to the dimension of the ambient manifold.
% \mainstructure*

% \begin{rem}\label{rem:only-1d-orbs}
%     If $\homeo$ acts on closed $M^3$ with only 1-dimensional orbits, then $M$ is equivariantly homeomorphic to $\Sigma \times S^1$ where $\Sigma$ is a closed surface and the action is trivial on the $\Sigma$ coordinate and evaluation on the $S^1$ coordinate.
% \end{rem}

Finally, we give a construction providing uncountably many non-conjugate examples of $\homeo$ actions on Seifert-fibered spaces of the form $(\Sigma, (2, 1), \dots, (2, 1))$ (including trivial circle bundles) or $\SO$ manifolds with the same restrictions.

% Motivated by this construction and other ad hoc positive and negative examples, we present the following conjectural picture for actions with only 1 and 2-dimensional orbits.

% \begin{conj}\label{conj:one-and-two-d-orbits}
%     Suppose that $M$ is an $\SO$ 3-manifold with base orbifold $\Sigma$, and $K\subseteq\Sigma$ be a closed set which avoids cone points and the boundary, then $M$ admits an action of $\homeo$ with all fibers over $K$ being $\homeo$ orbits and all orbits over the complement of $K$ being 2-dimensional if and only if:
%     \begin{enumerate}
%         \item $\Sigma - K$ is a disjoint union of topological disks and annuli
%         \item The components of the frontiers of all annuli are locally connected and all boundary components are contained in some topological annulus.
%         \item In every disk, the number of inaccessible prime ends plus the number of cone points is less or equal to 2
%         \item For all prime ends $e$ of each disk, $X(e) = Y(e)$
%     \end{enumerate}
%     Moreover, all actions on closed 3-manifolds with only 1 and 2-dimensional orbits arise in this way.
% \end{conj}
\begin{prop}\label{prop:lc-fr-construction}
    Suppose $(S_b, 2, \dots, 2)$ is a compact 2-dimensional orbifold with $b$ boundary  components and $\{X_k\}$ a collection of disjoint open topological disks, annuli and  M\"{o}bius bands in $S_b$ such that:
    \begin{enumerate}
        \item Each annulus and M\"{o}bius band contains no cone points and each disk contains at most 2 cone points. All boundary components are contained in the boundary of some annulus.
        \item The topological frontier of each disk and annulus is locally connected
    \end{enumerate}
    Let $M$ be the 3-manifold with the faithful $\SO$ action with orbit space $S_b$, multiplicity 2 exceptional orbits over the cone points and special exceptional orbits over the boundary components. Then, there are uncountably infinitely many non-conjugate extensions of the $\SO$ action on $M$ to an action of $\homeo$ on $M$ with a unique invariant tubular neighborhood over each $X_k$ and 1-dimensional orbits over every point of $S_b - \cup X_k$.
\end{prop}

This construction is easily modified to include $\pconf[S^1]{3}$ and $\conf[S^1]{3}$ orbits.
Note that as a consequence of \cref{thm:3mfld-global-top} and \cref{thm:weak-o-t-strat}, any $\homeo$ action on a closed 3-manifold with no global fixed points meets all hypotheses of this construction except (2.). Conjecturally, this construction is exhaustive in the situation when (2.) is met, but to understand this and the examples which are not provided by this construction, it is necessary to appeal to notions related to the prime ends compactifications of the $X_k$'s. For the details of the construction see \cref{subsec:details-of-construction} for a discussion of the prime ends technicalities, see \cref{subsec:prime-ends}

Finally, we note that allowing for global fixed-points leads to a somewhat subtler situation, which will be outside the scope of this paper. A small collection of observations concerning actions of $\homeo$ with global fixed points is gathered in \cref{subsec:fixed-points}.

\subsection{Proof of \texorpdfstring{\cref{thm:3mfld-global-top}}{Theorem \ref*{thm:3mfld-global-top}}}
To begin, we prove \cref{thm:3mfld-global-top}
\begin{proof}[Proof of  \cref{thm:3mfld-global-top}]
    Suppose $\rho: \homeo \to \homeo[M]$ is a nontrivial action without global fixed points. Note that $\homeo$ is simple, so the action is necessarily faithful, so in particular restricting to the rotation subgroup yields a faithful $\text{SO}(2)$ action on $M$. By the equivariant description of orbit types in \cref{prop:orbit-computation}, the only $\text{SO}(2)$ orbits with nontrivial isotropy are a single circle within a $\conf[S^1]{2}$ orbit with $\Z_2$ isotropy and a single circle within a $\conf[S^1]{3}$ orbit with $\Z_3$ isotropy. To understand the special exceptional orbits, note that by definition, each special exceptional orbit lies in an embedded torus consisting entirely of special exceptional orbits. Also, each special exceptional orbit is, in particular, a 1-dimensional $\SO[2]$ orbit with $\Z_2$ isotropy, so it must lie in a unique $\conf[S^1]{2}$ orbit. There are a number of ways to complete this proof, but perhaps the simplest is to take a continuous 1-parameter family of homeomorphisms which `tilts' the meridian circle of $\conf[S^1]{2}$. Then, the images of the torus containing the special exceptional orbits over this family is a continuous family of tori which meet at a single embedded $S^1$. Then, conclude this collection of $\conf[S^1]{2}$ orbits must be the desired tubular neighborhood.\fix{formalize this}
\end{proof}

Now we prove $\cref{thm:weak-o-t-strat}$ by appeal to \cref{prop:orbit-bundle-computation}.

\begin{proof}[Proof of \cref{thm:weak-o-t-strat}]
    Note the fixed point set of a point stabilizer subgroup $G_\theta$ is a closed set contained in $M_{(1)}$ and meets every one-dimensional orbit in a unique point. Since the ambient manifold is closed, the base orbifold is compact, so the projection map to the base orbifold is closed. So, the image of $\text{fix}(G_0)$ is a closed subset of the base orbifold; thus, its preimage under the projection map -- the set consisting of the one-dimensional orbits -- is closed. Trivially $M_{(3)}$ is closed, so $M - (M_{(1)}\cup \cl{M_{(3)}})\subseteq M_{(2)}$ is open. Then, in particular, each component of this set is equivariantly homeomorphic to one of the open tubular neighborhoods listed in \cref{prop:orbit-bundle-computation}. If there are no 3-dimensional orbits present (i.e., $M_{(3)}=\emptyset$), then this implies the hypotheses and first part of \cref{prop:lc-fr-construction} as mentioned in the preceding section.
\end{proof}

In the setting with only one-dimensional orbits, \cref{thm:3mfld-global-top} implies that $M$ is a circle bundle with every fiber invariant. The proof of \cref{thm:weak-o-t-strat} showed that the $\SO$ orbit space projection map restricted to $\text{fix}(G_0)$ is a homeomorphism onto the $\SO$ projection $M_{(1)}$; in particular, when all orbits are one-dimensional, the inverse of this restricted map gives a continuous section for the circle bundle structure. Trivially, the map from $M$ to $\Sigma \times S^1$  which maps $x$ to $(\pi_{\SO}, \theta)$ where $\theta\in S^1$ is the unique angle such that $x\in\text{fix}(G_\theta)$
is trivially a homeomorphism. With $\Sigma \times S^1$ equipped with the action $0\times \text{ev}$, this map is equivariant.

\subsection{Orbit compactification}
Motivated by the orbit-type stratification for proper Lie group actions on closed manifolds, it is useful to understand the topological closures of the non-compact orbits (i.e., 2 and 3-dimensional orbits). As a small point of notation, we recall the following definition from point set topology:
\begin{defn}
    The {\it frontier} of a set $X$, denoted $\text{fr}(X)$ is defined as $\cl{X}-X$
\end{defn}

The primary result of this section is the following:

\begin{thm}\label{thm:2d-orbit-cpctification}
    Suppose $X$ is a 2-dimensional orbit of any $\homeo$ action on a closed 3-manifold $M$, then $\text{fr}(X)$ has at most two components and each component consists of a single one-dimensional orbit or fixed point.
\end{thm}

More concretely, the following immediate corollary
may appeal to the reader:

\begin{figure}[b]
    \centering
    \includesvg[width=0.75\linewidth]{figures/2d orbit compactifications}
    \caption{(2D orbit closures) Blue marks 1D orbits and fixed points, black and grey are relief lines}
    \label{fig:2d-orbit-compactifications}
\end{figure}

\begin{cor}\label{cor:2d-orbit-closures}
    When $X$ is an annulus orbit, $\bar{X}$ is equivariantly homeomorphic to one of: $S^1\times [0,1]$, $T^2$, $\bar{D}^2$, $S^2$ or a ``pinched sphere'' ($S^2$ with its north and south-pole identified).

    When $X$ is a M\"{o}bius band orbit, $\bar{X}$ is an invariant closed M\"{o}bius band or $\RP^2$.

    In each case, the $\homeo$ action on the model space is the ``closed configuration space action.''
\end{cor}

Every structure listed above occurs in some action of $\homeo$ on a closed 3-manifold. Perhaps the most surprising entry on this list is the pinched sphere, which is not a manifold. The following provides an example of an action of $\homeo$ on $S^2 \times S^1$ having 2-dimensional orbits which compactify as pinched spheres.

\begin{eg}
    Consider the standard action $\rho$ of $\homeo[S^1]$ on $S^2$ (this action has two fixed points and a single annulus orbit). Then obtain an action $\rho':\homeo[S^1]\to \homeo[\cl{D^3}]$ given by constructing the trivial product action on $S^2\times [0,1]$ and collapsing one boundary sphere to a point. To get a closed example, equivariantly double this ball along its boundary. Finally quotient any symmetric closed interval of fixed points to a point, then each annulus orbit whose boundary was in this interval now has a pinched sphere closure.
\end{eg}


In addition to $\cref{thm:2d-orbit-cpctification}$, we prove the following closely related lemma about limits of sequences of 2-dimensional orbits.

\begin{lem}\label{lem:seq-of-annuli}
    Suppose $A_n$ is a sequence of two-dimensional orbits which converges in the sense of Kuratowski, then $\klim A_n$ is the closure of a two-dimensional orbit or a single invariant $S^1$
\end{lem}

This result is particularly notable as it provides a complete descriptions of the closures of $\homeo$ tubes about 2-dimensional orbits away from the ends of the fiber. More precisely, suppose $X$ is a $\homeo$ tube with fiber $F$ about some 2-dimensional orbit. 

\begin{cor}\label{cor:cpct-tube-boundary}
    Suppose $K$ is a compact subset of $F$, then $\cl{I(\pconf{2}\times K)} = \bigcup_{p\in K} \cl{I(\pconf{2}\times \{p\})}$
\end{cor}

Interestingly, with minor modification, the proofs in this section work for two-dimensional orbits of actions of $\homeo$ on any closed $n$-manifold. This is because any orbit in the frontier of a two-dimensional orbit must contain a $G_0$ fixed point and the action of $G_0$ on any orbit of dimension greater than one is fixed-point free, so higher orbit dimensions need not be considered. Note that when discussing the $\SO$ orbit space, while it need not be a manifold in higher dimensions, all that is needed is the fact that it is Hausdorff, which is generically true. This is a direct consequenc of the following stronger form of \cref{thm:2d-orbit-cpctification}, which can be stated as follows:

\begin{thm}
    Suppose $\homeo$ acts on closed $n$-manifold $M$ and $I$ is an interval orbit of $G_{\theta_1\dots\theta_k}$, then $\cl{I}\cong S^1$ or $[0,1]$ is the union of $I$ with one or two $G_{\theta_1\dots\theta_k}$ fixed points.
\end{thm}

This result is not used in this paper, but we present it as it suggests an inductive structure to prove a higher dimensional $\homeo$ orbit closure theorem.

The remainder of this section is devoted to proving \cref{thm:2d-orbit-cpctification}. The overall structure of the proof is to first show that the frontier of a two-dimensional orbit consists of only one-dimensional orbits and global fixed points, then to show that in each end of the annulus or M\"{o}bius band, there is a unique such one-dimensional orbit or fixed point. For convenience, we introduce the following convention for labelling the end(s) of a two-dimensional orbit:

\begin{defn}
    Let $A$ be an annulus orbit, fix a model annulus $f:A\to S^1\times I$ (because of this choice, the plus or minus labelling is not intrinsic). Then, denote by $\text{fr}_+(A)$ the set of frontier points which can be accumulated by sequences with increasing $I$ coordinate and $\text{fr}_-(A)$ analogously. Note that $\text{fr}_+(A)\cup\text{fr}_-(A) = \text{fr}(A)$, but they are not necessarily distinct. We can compatibly label subsets of the frontier of a $G_0$ invariant interval in an annulus $A$. For notational consistency, when $X$ is a M\"{o}bius band orbit, let $\text{fr}_+(X) = \text{fr}_-(X) = \text{fr}(X)$.
\end{defn}

For the following two proofs, we want to make use of a kind of canonical coordinates on interval orbits of $G_0$.

\begin{rem}\cite{chen:StructureTheorems}
    Suppose $A$ is an annulus orbit of $\homeo$ and $I\subset A$ is an interval orbit of $G_0$. Then, we can define two homeomorphisms $\varphi, \psi:(0,1) \to I$ by $\varphi(\theta) = \text{fix}(G_0\cap G_\theta)$ and $\psi(\theta) = \text{fix}(G_0\cap G_{-\theta})$. One of these maps will send monotonic sequences that converge to $1$ to sequences which converge to something in $\text{fr}_+(I)$, we will denote the inverse of this map as $f_I$, which we will refer to as the {\it canonical homeomorphism between $I$ and $S^1 - \{0\}$}. Note, the same construction works for the one interval orbit in a M\"{o}bius band orbit.
\end{rem}

With these notions in hand, we are ready to prove the first step of the outline.

\begin{prop}
    Suppose $X$ is an 2-dimensional orbit, then $\text{fr}(X)$ consists of 1-dimensional orbits or fixed points.
\end{prop}

\begin{proof}
    Trivially, $\text{fr}(X)$ contains no 3-dimensional orbit.

    Suppose toward a contradiction that $\text{fr}(X)$ contains some 2-dimensional orbit $X'$. Now, consider the restriction of the $\homeo$ action to the point stabilizer subgroup $G_0$. There is some interval orbit $I$ in $X$. Since $\pi_{SO(2)}$ is continuous and closed, $$\pi_{SO(2)}(\bar{I}) = \overline{\pi_{SO(2)}(I)} = \overline{\pi_{SO(2)}(A)} = \pi_{SO(2)}(\bar{A}).$$ In particular, this means that $\text{fr}(I)$ must nontrivially intersect $X'$. The closure of every $G_0$ orbit in $X'$ contains at least one interval orbit $I'$, so $I'\subseteq \text{fr}(I)$. In particular, there is some sequence of points $\{x_n\}\subset I$ converging to a point $x\in I'$.

    From the remark, there are continuous bijections $f:I\to S^1-\{0\}$ and $g:I'\to S^1-\{0\}$ given by mapping the unique fixed point of $G_\theta\cap G_0$ on $I$ to $\theta\in S^1 - \{0\}$ and similarly for $I'$. Then, up to a subsequence, $\{f(x_n)\}$ is a monotonic sequence converging to 0. So we can choose a monotonic element $\phi\in G_0$ satisfying $\phi(f(x_k)) = f(x_{k+1})$. In particular, we must have that $\{\rho(\phi)( x_n)\}$ converges to $x$, but by monotonicity, $\rho(\phi)(x) \neq x$, violating continuity, which is a contradiction. Thus $\text{fr}(A)\cap X = \emptyset$
\end{proof}

The next step is a restatement of the main result of this section using the notation of the preceding proof.

\begin{prop*}[Restatement of \cref{thm:2d-orbit-cpctification}]
    Suppose $X$ is a 2-dimensional orbit. Then, $\text{fr}_+(X)$ and $\text{fr}_-(X)$ are (not necessarily distinct) one-dimensional orbits or fixed points.
\end{prop*}

\begin{proof}
     Let $I$ be a $G_0$-invariant interval in $X$. Note that by the above, $\text{fr}_+(I)$ consists of some union of invariant $S^1$'s and $G_0$ fixed points. A rather convenient property of $\homeo$ is that there are elements of $G_0$ taking any monotonic sequence in $I$ to any other monotonic sequence in $I$. In particular, if $p$ is any $G_0$ fixed point in $\text{fr}_+(I)$, then by taking a convergent monotonic subsequence of any sequence of points in $I$ converging to $p$, then in fact $\text{fr}_+(I) = p$. So, in particular, the frontier of $X$ consists of the fiber over a single point in the $\SO$ orbit space, which is either an $S^1$ or a global fixed point.
\end{proof}

To observe \cref{cor:2d-orbit-closures}, it suffices to note that there is an equivariant quotient map from the closed annulus or closed M\"{o}bius band with the closed configuration space action to $\cl{X}$ which is a homeomorphism on the interior. The only permissible quotients of this form are produced collapsing invariant $S^1$'s to points, gluing invariant $S^1$'s together by the identity or gluing global fixed points together.

Finally,
\begin{proof}[Proof of \cref{lem:seq-of-annuli}]
    Let $I_n$ be a sequence of $G_0$ interval orbits chosen from each $A_n$. By a result on pg. 339 of Kuratowski's book \cite{kuratowski:TopologyI}, passing to subsequences does not change the limit. Leveraging the compactness of $M$ and the generalized Bolzano-Weierstrass Theorem, pass to a subsequence such that $I_n$ is K-convergent, there are convergent sequences in the $I$'s $(c_n)\to c$, $(x^+_n)\to x^+$ and $(x^-_n)\to x^-$ with $p(c_n)$ constant, $p(x^+_n)$ unbounded monotonic increasing and $p^(x^-_n)$ unbounded monotonic decreasing. By \cref{lem:union-over-cpct}, any bounded convergent sequence chosen from the $I_n$'s converges to a $G_0$ translate of $\text{lim} c_n$. So in particular the limit any bounded convergent sequence is realized as the limit of some constant convergent sequence. There is an angle $\theta$ such that $\pstab{0, \theta}$ fixes this entire sequence, so the limit point must be fixed by  $\pstab{0, \theta}$, so the containing $G_0$ orbit is an interval orbit or $G_0$ fixed point. Suppose $(x_n)$ is some unbounded convergent monotonic increasing (or decreasing, resp.) sequence. Further suppose that $X$ is a finite subset of $S^1$ such that $x^+$ is fixed. There is an infinite tail of both sequences strictly greater than the set $X$, moreover there are elements of $\pstab{X}$ taking any monotonic unbounded sequence which is bounded below by $X$ to any other such sequence. In particular, it is there is an element of $\homeo$ taking some infinite tail of $(x^+_n)$ to some infinite tail of $(x_n)$ and fixes $x^+$. Thus, $\text{lim}(x_n) = x^+$. Since every element of $G_0$ preserves unboundedness and monotonicity, $x^+$ and $x^-$ are $G_0$ fixed points. Finally, since every sequence chosen from the $I_n$'s has a bounded or monotonic unbounded subsequence and passing to subsequences preserves limits, $\klim I_n = \{x^+, x^-\}\cup \rho(G_0)\cdot(c)$. In a compact Hausdorff space, K-limits of connected sets are connected, so if $c$ is a $G_0$ fixed point, $c=x^+=x^-$, in particular, $\klim I_n$ is a single point. Additionally, K-limits are closed, so if $\rho(G_0)\cdot(c)$ is an interval orbit, then $\klim I_n$ must include its frontier points, and trivially these must be $x^+$, $x^-$. Finally, by $\cref{lem:union-over-cpct-k-lim}$, $\klim A_n$ is the image under $\SO$ of $\klim I_n$, i.e. a one-dimensional orbit or the closure of a two-dimensional orbit.
\end{proof}

\subsection{Details of the construction \texorpdfstring{\cref{prop:lc-fr-construction}}{Proposition \ref*{prop:lc-fr-construction}}}\label{subsec:details-of-construction}
What follows provides the details of our construction for a large family of $\homeo$ actions on closed 3-manifolds with only one and two-dimensional orbits.
\begin{proof}[Proof of \cref{prop:lc-fr-construction}]
    The overall structure of the construction is as follows. Note that for convenience we denote the (necessarily closed) $S_b - \cup X_k$ as $K$.
    For each disk, annulus or M\"{o}bius band $U\in\{X_k\}$, to construct the desired action over $U$, we start with model $\homeo$ actions on $\SO$ manifolds with boundary which restrict to a model tube action on the interior, make some free choice of modifications on the boundary and fix a homeomorphism to the prime ends compactification of $U$ and pull back the model $\SO$ action. The local-connectedness of the frontier provides a continuous extension of the inclusion of $U$ in $\hat{U}$ to a surjective continuous map from the prime ends compactification of $U$ to $\bar{U}$\cref{thm:lc-fr-prime-ends}. Finally, we use this map to construct an equivariant gluing from the model action to the corresponding frontier component of $K\times S^1$. The two choices made in this construction were in modifying the model action and choice of homeomorphism from the model action base space to the prime ends compactification of $U$. While differing choices here may result in globally conjugate actions, there is a highly infinite family of choices which can be made which result in pairwise nonconjugate actions.

    We begin the construction simply by letting $\homeo$ act on $K\times S^1$ trivially on the $K$ factor and by evaluation on the $S^1$ factor. Let $U$ be a fixed annulus, M\"{o}bius band or disk in $S_b - K$. Note that since we assumed $\lvert K\rvert > 2$, it is not possible that the prime ends compactification of $U$ is closed, so it must have nonempty boundary. In particular, $\hat{U}$ is homeomorphic to a closed disk, closed annulus or closed M\"{o}bius band. Let $\cl{B}$ be the unique model closed $\homeo$ invariant tubular neighborhood over $\hat{U}$. Choose any family of intervals $\mathscr{I}\coloneqq \{I_\alpha\}$ in the boundary of $\pi_{\SO}(\cl{B})$ such that the quotient of the boundary by the equivalence relation having $\mathscr{I}$ as equivalence classes is homeomorphic. By a Bing shrinking argument, the quotient corresponding to a collection of intervals is homeomorphic to a circle if and only if no sequences of intervals converges to an endpoint of another interval in the collection.

    By identifying the $S^1$'s over each $I_\alpha$ along standard $\homeo$ coordinates, the result is an $\SO$ isomorphic quotient of $\cl{B}$. Fix some homeomorphism $\varphi: \pi_{\SO}(\cl{B}) \to \hat{U}$. By pulling back the $\SO$ along this homeomorphism, we consider $\cl{B}$ as a $\SO$-manifold over $\hat{U}$. Since, $\fr{U}$ is locally connected, there is a continuous (surjective) extension of the inclusion of $U$ to a map $f:\hat{U}\to \cl{U}$\cite{mather:TopologicalProofs}. Define a map $\tilde{f}:\partial\cl{B}\to\pi_{\SO}^{-1}(\fr{U})$ by identifying each $\SO$ fiber over a point in $\partial\hat{U}$ with the fiber over $f(p)$ along $\homeo$ coordinates. This is a product of continuous, surjections so it is, in particular, a continuous surjection. A continuous surjection from a compact space to a Hausdorff space is necessarily a quotient map and this quotient is definitionally compatible with the action, so this gives an equivariant gluing. Thus, the result is an action of $\homeo$ on a $\SO$-manifold with base space $K\cup U\subset S_b$

    This construction can be done simultaneously for all components of $S_b - K$, and results in an action with the desired properties over all of $S_b$. In general, choosing different collections of intervals and different homeomorphisms to the prime ends compactification of $U$ will result in non-conjugate $\homeo$ actions.
\end{proof}

\subsection{Exotic prime ends examples}\label{subsec:prime-ends}
The primary goal of this section is to impress upon the reader that the assumption of locally connected frontier in \cref{prop:lc-fr-construction} is not sharp, and it is possible to relax it slightly by venturing into prime ends theory. We will first demonstrate that there are still unusual examples provided the local connectivity assumption. Later, we will give examples which suggest what the sharper condition should be. Note that all that follows requires passing familiarity with the topological theory of prime ends. For convenience, we have included a short primer on the fundamental notions following \cite{mather:TopologicalProofs}. 

Let $\Sigma$ be a closed surface and $U\subset \Sigma$ an open subset with $H_1(U, \Z_2)$ finite. Our goal is to compactify $U$ as a surface with boundary $\hat{U}$ in a principled way such that in nice situations, there is a quotient map from $\hat{U}\to\cl{U}\subseteq\Sigma$. 

\begin{defn}
    A topological chain is a descending sequence $V_1\supset V_2\supset\dots$ of open connected subsets of $U$ whose frontiers in $U$ are nonempty and connected and $\cl{\fr{U_i}}\cap\cl{\fr{U_j}}=\emptyset$ for $i\neq j$, where closures are taken in $\Sigma$.

    A chain $\tau=\{V_1\supset V_2\supset\dots\}$ is said to divide $\sigma =\{W_1\supset W_2\supset\dots\}$ if every $W_i$ is contained in some $V_j$. Two chains are equivalent if they divide one another. A chain is said to be prime if every chain which divides it is equivalent to it.
\end{defn}

The set of prime chains up to equivalence (called prime points) admits a very natural topology. Moreover, $U$ embeds into the space of prime points by taking, e.g., a decreasing sequence of disks about a point. Those prime points which are not equivalent to chains of disks about points in $U$ are called prime ends. Every prime end interacts with the topological frontier of $U$ in some way:

\begin{defn}
    Suppose $e$ is a prime end and $\{V_1\supset V_2\supset\dots\}$ is any representative chain. The impression of $e$ is $Y(e)\coloneqq \cap_i \cl{V_i}$. If there is a chain $\{V_1'\supset V_2'\supset \dots \}$ whose frontiers Hausdorff limit to a point $x$, $x$ is said to be a principal point for $e$, and the principal set $X(e)$ is the collection of all principal points for $e$.
\end{defn}

In general, $X(e)\subset Y(e)$, and they are always connected, compact and nonempty subsets of $\fr{U}$. When $X(e)$ is reduced to a single point, $e$ is said to be accessible. Accessibility is equivialent to the existence of a path $\gamma(t):(0,1)\to U$ such that $\lim\limits_{t\to 1}\gamma(t)$ exists and lies in $X(e)$. 

\begin{thm}\label{thm:lc-fr-prime-ends}
    $\fr{U}$ is locally connected if and only if $Y(e)$ is a singleton for all prime ends $e$. When $\fr{U}$ is locally connected, $Y:\hat{U}\to \bar{U}$ is continuous and surjective.
\end{thm}


\begin{figure}
    \centering
    \includesvg[width=0.5\linewidth]{figures/admissible hedgehog}
    \caption{Admissible hedgehog space complement}
    \label{fig:admissible-hedgehog}
\end{figure}

\begin{eg}[Admissible Hedgehog Complement]
    Consider the following modified set described in \cite{epstein:PrimeEnds}:
    $$K\coloneqq\{ (r, \theta) \vert \theta = 2\pi p/2^n,\, n\in \Z^+,\, p \text{ odd},\, 0 < p < 2^n,\, 0\leq r\leq 1/2^n\}\cup(S^2-B_1(0))\subseteq S^2$$
    where $S^2$ is considered as $\R^2$ union a point at infinity. Topologically, $K$ is closed and $S^2 - K$ is a disk; moreover, $K$ is locally connected. Thus, \cref{prop:lc-fr-construction} provides actions of $\homeo$ on $S^2\times S^1$ with a $\pconf{2}\times S^1$ tubular neighorhood, the $\SO$ projection of which is the complement of $K$. Clearly any homeomorphism of $S^2\times S^1$ can only cyclically permute the radial segments in $K$ cross $S^1$. In particular, the cyclic sequence of number of $S^1$ fibers per radial segment which occur as the frontier of infinitely many annulus orbits is a global homeomorphism invariant, so the cardinality of the set of nonconjugate actions arising from this construction is at least $2^\N$.
\end{eg}

For the remainder of this section, if there exists an action of $\homeo$ on a closed 3-manifold with a $\homeo$ invariant tubular neighborhood over a particular embedded disk or annulus $U$ in the base space, we will call it `admissible,' if there is no such action we will call it `inadmissible.' The first example will be an inadmissible embedded disk with an inaccessible prime end with $Y(e)\neq X(e)$. Next will be an admissible modification of the preceding example such that $X(e) = Y(e)$ for the inaccessible prime end. Finally, we will provide an example of an inadmissible embedded annulus with an accessible prime end with $Y(e)\neq X(e)$.

A classical family of examples from prime ends theory comes from considering the complements of variations on the comb space. First we will provide an inadmissible example, which we will then modify slightly to get an admissible example depicted in \cref{fig:inadmissible-comb} and \cref{fig:admissible-comb}. In both cases, the disk under discussion is the complement of the set $K$ in black. The dotted lines show the $\SO$ projections of annulus orbits.

\begin{eg}[Inadmissible Comb Complement]
    Let the set $K$ be the union of the boundary of the unit square in $\R^2$ with the set of ascending vertical segments of length 2/3 based at all points $(\frac{1}{2n+1}, 0)$ and descending vertical segments of length 2/3 based at all points $(\frac{1}{2n}, 1)$. Let $U$ denote the bounded component of $\R^2 - K$. While our claim here will be that a particular action on $U\times S^1$ does not extend to $K\times S^1$, it is not difficult to show by a similar but more careful argument that no orbit bundle action on $U\times S^1$ can extend to the frontier. The action of interest is given by applying the construction in \cref{prop:lc-fr-construction} to $U$ considered as a subset of $\R^2 - \R^2_{y\leq 0}$ such that the $\SO$ projections of annulus orbits are horizontal between $y=1/3$, and $y=2/3$ and radial based at the end points of the vertical segments as depicted in \cref{fig:inadmissible-comb}. The primary obstruction to extending the action to the $y$-axis is sequences of orbits like the one marked in red. Since the fibers over all points in $K-\{0\}\times\R$ are invariant, there are sequences of 1-dimensional orbits converging to all points of $K$ in the $y$-axis, so in particular, any extended action must have all fibers over points on the $y$-axis invariant, but by \cref{lem:seq-of-annuli}, the limit set of the red sequence of annuli must be a closed 2-dimensional orbit, which is a contradiction.
\end{eg}

\begin{figure}
\centering
\begin{minipage}{.5\textwidth}
  \centering
  \includesvg[width=.75\linewidth]{figures/inadmissible comb}
  \caption{Inadmissible comb complement}
  \label{fig:inadmissible-comb}
\end{minipage}%
\begin{minipage}{.5\textwidth}
  \centering
  \includesvg[width=.75\linewidth]{figures/admissible comb}
  \caption{Admissible comb complement}
  \label{fig:admissible-comb}
\end{minipage}
\end{figure}

In the preceding example, we gave an obstruction for extending an action, and we will show that it was, in some sense, the only obstruction.

\begin{eg}[Admissible Comb Complement]
    Let $K$ be constructed identically, but with the lengths of the vertical segments converging to 1. We also slightly modify the action for convenience such that the region in between the two green curves cross $S^1$ is clearly $\homeo$ isomorphic to $[0, 1]\times (0,1]\times \times S^1$ with the action $0\times \rho_{K, \lambda}$ with $K=\{1/n\}$ and $\lambda$ taking the value $0$ on even intervals and $1$ on odd intervals. The action $\rho_{K, \lambda}$ extends to an action on $[0,1]\times S^1$ with $\{0\}\times S^1$ being a 1-dimensional orbit, so we extend the product action continuously to the $y$-axis. To show that this gives a continuous extension of the action over $\cl{U}-\{0\}\times \R$, it remains to check continuity at $(0,0)\times S^1$ and $(0,1)\times S^1$, but this can be done explicitly and is of little interest.
\end{eg}

Both examples have an inaccessible prime end corresponding to all of the frontier points on the $y$-axis. The key difference is that in the first example, only the points between $y=1/3$ and $y=2/3$ are principle, while in the second example all points on the $y$-axis are principle. Note that by a very similar method, it is also possible to construct an action which projects to the complement of the unit circle union a spiral converging to the unit circle (see \cref{fig:admissible-spiral}). This example also has one inaccessible prime end with $X(e)=Y(e)$. The details are nearly identical, so they will be suppressed. Next, we will exhibit an example where all prime ends are accessible, but by virtue of there being a prime end with $X(e)\neq Y(e)$, the disk is inadmissible.

\begin{eg}[Inadmissible Topologist's Sine Curve Complement]
    Let $K$ be the union of the closed topologist's sine curve (truncated on the right) and the unit circle (see \cref{fig:inadmissible-top-sin}). Let $U$ be the bounded component of $\R^2 - K$. Our claim is that there is no action of $\homeo$ on $U\times S^1$ which extends to an action on $\cl{U}\times S^1$. Suppose $\homeo$ acts on $U\times S^1$ with all two-dimensional orbits (and is thus exactly the $\pconf{2}\times S^1$ action). If this action were to extend, every fiber over the sine curve must be an $S^1$ orbit. Then, there are sequences of $S^1$ orbits which limit onto every fiber in the vertical interval $(\{0\}\times [-1, 1])\times S^1$, so every fiber over this interval must be an orbit. The key observation now is that for purely topological reasons, if $(m_n)\to (0,-1)$ is the sequence of minima of the topologist's sine curve, then there must be a sequence of annulus orbits $(A_n)$ such that (WLOG) $(\pi_{\SO}(\text{fr}_{+}(A_n))))$ is a subsequnce of $(m_n)$ and the curve $\pi_{\SO}(A_n)$ leaves a neighborhood of the sine curve. Then, pass to a convergent subsequence of the $A_n$'s. We immediately obtain a contradiction, since $\klim{A_n}$ must contain the entire vertical interval $(\{0\}\times[-1, 1])\times S^1$.
\end{eg}
\begin{figure}[t]
\centering
\begin{minipage}{.5\textwidth}
    \centering
    \includesvg[width=0.75\linewidth]{figures/admissible spiral}
    \caption{Admissible spiral complement}
    \label{fig:admissible-spiral}
\end{minipage}%
\begin{minipage}{.5\textwidth}
    \centering
    \includesvg[width=0.75\linewidth]{figures/inadmissible top sin}
    \caption{Inadmissible topologist's sine complement}
    \label{fig:inadmissible-top-sin}
\end{minipage}
\end{figure}

These examples suggest the following conjecture:
\begin{conj}\label{conj:one-and-two-d-orbits}
    Suppose that $M$ is an $\SO$ 3-manifold with base orbifold $\Sigma$, and $K\subseteq\Sigma$ be a closed set which avoids cone points and the boundary, then $M$ admits an action of $\homeo$ with all fibers over $K$ being $\homeo$ orbits and all orbits over the complement of $K$ being 2-dimensional if and only if:
    \begin{enumerate}
        \item $\Sigma - K$ is a disjoint union of topological disks and annuli
        \item The components of the frontiers of all annuli are locally connected and all boundary components are contained in the frontier some annulus.
        \item In every disk, the number of inaccessible prime ends plus the number of cone points is less or equal to 2
        \item For all prime ends $e$ of each disk, $X(e) = Y(e)$
    \end{enumerate}
\end{conj}

\subsection{Remarks on \titlesafehomeo actions with global fixed points}\label{subsec:fixed-points} 
Of the results proven in the preceding sections, only \cref{thm:2d-orbit-cpctification} takes into account actions with global fixed points. This section presents a brief summary of the theory of $\SO$-manifolds with fixed points and provides a construction for a large family of $\homeo$ actions with global fixed points. 

Note that any global fixed point of a $\homeo$ action is fixed by the rotation subgroup by definition. Moreover, by \cref{prop:orbit-computation}, within a nontrivial $\homeo$ orbit, the rotation subgroup acts without fixed points, so the $\SO$ fixed points are exactly the fixed point of the overall $\homeo$ action. 

\begin{eg}\leavevmode
    \begin{enumerate}
        \item Let $\rho_F$ be an action of $\homeo$ on the disk $D^2$ which fixes the center point has a small disk $D_\varepsilon$ worth of $S^1$ orbits centered on $C$. Note outside $D_\varepsilon$ we place no restrictions on $\rho_F$. The action $\rho_F \times 0$ on $V_F \coloneqq D^2\times S^1$ has a circles worth of global fixed points $\{C\}\times S^1$. Equivariantly doubling along the boundary gives an action on $S^2\times S^1$ with two circles worth of global fixed points.

        \item Let $U$ be $D^2 - \{C\}\cup\partial D^2$. Clearly $U$ is homemorphic to an open annulus, and $\fr{U}$ is locally connected, so \cref{prop:lc-fr-construction} yields an action on $V= D^2\times S^1$ with a $\homeo$ tube of type $\pconf{2}$ over $U$ and $S^1$ orbits over $C$ and $\partial D^2$. If in the preceding example instead of doubling, we glued by the map which takes the $\partial D^2 \times \{\theta\}$ to $\{p\}\times S^1$ in the boundary of $V$, the resulting space is homeomorphic to $S^3$. This gluing preserves the action as it clearly conjugates the actions on the boundary.
        
        \item If we again start with $V_F$ and glue by the same map to the $(\D^2, (2,1))$ $\homeo$-tube compactified in the natural way, we obtain an action on $L(2, 1)$ with global fixed points. (note the fact that the order of the Seifert invariants is not reversed for the lens space is not a mistake, see the proof of Theorem 3 in \cite{orlik:ActionsSO2}).
    \end{enumerate}
\end{eg}


Each of these examples have circle of fixed points sitting over boundary components of the $\SO$ orbit space. By appealing to \cite{orlik:ActionsSO2}, we can observe that this is in fact a general fact which comes from considering the action of the rotation subgroup. Additionally, Orlik and Raymond give an $\SO$-equivariant connect sum formula for all closed 3-manifolds admitting faithful $\SO$ actions with global fixed points. In particular, they classify $\SO$ manifolds in terms of invariants $(b; (\varepsilon, g, \bar{h}, t); (\alpha_1, \beta_1),\dots, (\alpha_n, \beta_1))$
where $b$ is the integer part of the Euler number, $\varepsilon = o\text{ or }\bar{n}$ to denote orientability of the base surface, $g$ denotes its genus, $\bar{h}$ records the number of components of the fixed point set, $t$ records the number of components of special exceptional orbits and finally $(\alpha_i, \beta_i)$ function as the classical Seifert invariants. 

Using the preceding examples, as well as an analgous construction for $\RP^2\times S^1$, we obtain an infinite family of non-conjugate actions of $\homeo$ with global fixed points on all spaces given by invariants $(0; (\varepsilon, g, \bar{h}, t); (2, 1),\dots (2, 1))$ parameterized by the set of all actions of $\homeo$ on the closed annulus. This works since these examples are exactly the connect summands in part (1) and (2) of Theorem 3 in \cite{orlik:ActionsSO2}. These connect sums can be formed $\homeo$-equivariantly by taking balls contained within $D_\varepsilon \times S^1$ which meet the circle worth of fixed points in a closed interval and gluing maps which map fixed points to fixed points. We can also trivially observe that $\SO$ actions on 3-manifolds with $\alpha_i > 3$ cannot extend to $\homeo$ actions. 

% Two immediate, fundamental (and we believe approachable) questions are
% \begin{question}\leavevmode
%     \begin{enumerate}
%         \item If a manifold admits a particular $\SO$-equivariant connect sum formula and a $\homeo$, can it always be decomposed as a $\homeo$-equivariant connect sum?
%         \item Which $\SO$ actions with fixed points extend to $\homeo$ actions?
%     \end{enumerate}
% \end{question}
% Toward the second question, if $M$ is an $\SO$ manifold with only $\alpha=2$ for all lens space factors in the connect sum, then there is an explicit construction providing extensions of the $\SO$ action to infinitely many non-conjugate $\homeo$ actions. 


% \subsection{Orbit Bundle Compactification}



% \begin{prop}\label{prop:ann_prime_ends_nice}\improve{need to comment on what I mean by "prime end associated to" because this can be made precise.}
%     All prime ends associated to a frontier of an interior 2D orbit are of the {\it first kind}, i.e. they are accessible and $X(e) = Y(e)$.
% \end{prop}

% \begin{proof}
%     Let $A$ and $\sigma$ the frontier component of interest.
%     There are several cases that need to be treated separately here:
%     {\it \bf $A$ annulus}

%     {\it $A$ has 2-sided approx} Let $X_n$ and $Y_n$ be sequences of annulus orbits such that $X_n$ and $Y_n$ approach $A$ from different sides in the fiber, $\klim X_n = \klim Y_n = \cl{A}$. Let $\sigma^X_n$ and $\sigma^Y_n$ be the frontier components of $X_n$ and $Y_n$ respectively which limit onto $\sigma$. That the $\sigma^X_n$ (and $\sigma^Y_n$) are all distinct is what is meant by `has 2-sided approx.' Let $U_n$ be the open intervals worth of 2d orbits between $X_n$ and $Y_n$ containing $A$. Then, define $V_n = \pi_{\SO}(U_n\cap p^{-1}(\{(\theta, \varphi)\,\vert \,\lvert\theta - \varphi\rvert < 1/2^n\}))$. It is easy to verify that this is a topological chain in the sense of Mathers \cite{mathers}. Moreover, $\bigcap \cl{\pi_{\SO}^{-1}(V_n)} \subseteq \bigcap \cl{U_n} = \cl{A}$ and trivially does not contain any points of $A$ or the other frontier (if it is distinct). So the impression of this chain is reduced to a single point, thus the chain is prime and the corresponding prime end satisfies $X(e) = Y(e)$.

%     {\it $A$ has 1-sided approx}
%     If there exists no such sequence $X_n$ on one side of the annulus orbit $A$, but there is a sequence $Y_n$ on the other side, then there are finitely many annulus orbits on that side of $A$ which have distinct frontiers. By a connectedness trick, we can see that the frontiers of all annulus orbits on that side of $A$ must coincide with the frontier of $A$. If the $\SO$ base space is an annulus, we fall through to the next case, so assume WLOG the base space is a disk. If there is a M\"{o}bius band orbit on this side of $A$, then both frontiers of $A$ must coincide with the frontier of that M\"{o}bius band orbit, so we defer this to the final case at the bottom. Otherwise, take a K-convergent sequence of annulus orbits which goes infinity in the fiber on this side of $A$. If this sequence limits onto the closure of a 2D orbit, then its frontier must coincide with the frontier of $A$. If it limits onto a 1D orbit, then it must be $\sigma$ and the frontier of $A$ is connected. In the first case, the prime chain we will construct is essentially identical to the chain we constructed in the {\it 2-sided approx} case. In the last case, we will take as our $V_n$'s the $\SO$ projection of the side of $Y_n$ which contains $A$.

%     {\it $A$ has no approx}
%     This case contains finitely many examples up to conjugacy, must be checked manually.

%     {\it\bf $A$ M\"{o}b}
%     This will be essentially the same trick, but we will use both frontiers of a sequence of annulus which converges to the M\"{o}bius band. It will require the same case structure, but for the two ends of the annulus orbits that converge onto the M\"{o}bius band. \improve{much of the proof here is hand-wavey}
% \end{proof}

% Note, clearly for an orbit bundle $B$ every point in $\text{fr}(B)$ is in the limit of some sequence of 2D orbits (just by applying the generalized Bolzano-Weierstrass theorem)

% \begin{prop}
%     If $B$ is maximal, there is a single prime end associated to each end of the fiber, and $X(e) = Y(e)$
% \end{prop}

% \begin{proof}
%     There is a similar subtle case structure as to the preceding result. The simplest case is that there is a sequence of annuli going toward a particular end with pairwise distinct frontiers. Since the orbit bundle is assumed to be maximal, the limit of these annuli is a single 1D orbit. In a compact Hausdorff space, convergence in the sense of Kuratowski and convergence in the sense of Hausdorff are equivalent, so by Corollary 4 \cite{mathers}, the chain given by the components of the complement of the annuli which contains the end of interest is prime. Since every frontier $S^1$ orbit of the orbit bundle in the direction of a given end is the limit of some sequence of annuli, and the chain associated to any sequence of annuli which goes to a particular end clearly divides any other such chain. So, there is a unique prime end associated with a particular end and $X(e) = Y(e)$ by definition.

%     If there is no such sequence of annuli, then for any sequence of annuli in the direction of that end, there are only finitely many distinct frontier $S^1$'s, so any such sequence is eventually constant in one of its frontiers. By passing to a subsequence of these annuli which converges and which has at least one of its frontiers constant. But, since the orbit bundle is maximal, the limit of this sequence must consist of a single $S^1$ orbit which clearly must be the same as the constant frontier(s) in the sequence. Thus, this is accounted for by \cref{prop:ann_prime_ends_nice}.
% \end{proof}

% Trivially, in a given orbit bundle, the only prime ends which could be inaccessible are the ones associated to ends of the fiber, and the number of ends of the fiber plus the number of cone points in an orbit bundle with disk base space is less than equal to 2. Thus these results together imply only if of \cref{conj:one-and-two-d-orbits}.

% \section{\texorpdfstring{$\PSL$}{PSL(2, R)} Acting on Surfaces}
% \fix{this is probably a dissertation chapter, not a paper section}

\section{Future Work}\label{sec:future-work}
The two primary themes for future work are providing a complete description of $\homeo$ actions on closed 3-manifolds and studying other specialized homeomorphism group action settings. Under both themes, we provide several starting points or questions which we feel are answerable using similar techniques to those in this paper.

\subsection{Completing Description of \texorpdfstring{$\homeo$}{Homeo\_0(S\string^1)} Acting on Closed 3-manifolds}
\begin{figure}[b]
    \hspace{5em}
    \centering
    \includesvg[width=0.6\linewidth]{figures/pointstabilizeraction}
    \caption{Action of $G_0$ on $\pconf[S^1]{2}$ orbit and action of $G_0\cap G_\theta$ on $G_0$ disk orbit}
    \label{fig:pointstabilizersubgroup}
\end{figure}
Our three main goals under this theme are to prove a similar result to \cref{thm:2d-orbit-cpctification} for 3-dimensional orbits, prove \cref{conj:one-and-two-d-orbits} and to extend the preceding and all results in the paper to include the $\homeo$ actions with global fixed points. To study the compactifications of 3-dimensional orbits, the first key insight is to note that the proof of \cref{thm:2d-orbit-cpctification} can be viewed as a property of the closures of interval orbits of (multiple) point stabilizer subgroups. To prove \cref{conj:one-and-two-d-orbits}, we consider prime ends associated to the frontiers of 2-dimensional orbits and prime ends associated to ends of the fiber separately --- in both cases \cref{lem:seq-of-annuli} will be useful. Finally, to extend our results to actions with global fixed points, we show that global fixed points for a $\homeo$ action are exactly the global fixed points of the rotation subgroup action, we can then appeal to the full equivariant classification of $\SO$ manifolds in \cite{orlik:ActionsSO2}.

% The action of $G_0$ on a $\pconf{3}$ orbit of $\homeo$ decompose it into three 3-ball orbits and three disk orbits. Restricting to $G_{0,\theta}$ further subdivides each disk into three disk orbits and two interval orbits. Allowing $\theta$ to vary over $S^1-\{0\}$ gives two transverse product structures on each disk orbit in terms of the interval orbits.
% The first goal of this direction would be to prove an analogous result to the previous remark for these disk orbits by using this remark and \cref{lem:union-over-cpct}. This structure also suggests an inductive structure to prove a similar result for $n$-dimensional $\homeo$ orbits.

% Suppose $X$ is a $\homeo$ tube with fiber $F$. Let $x$ be some frontier point of the base surface of $X$. Let $(x_n)\subset X$ be a sequence converging to $x$. If the collection of orbits containing $(x_n)$ meets the fiber in a precompact set, then $x\in\cl{A}$ for some 2-dimensional orbit $A\subset X$.

% Otherwise, we pass to a subsequence of $(x_n)$ which is monotonic in the fiber. In particular, the frontier of an orbit bundle can then be decomposed into points which lie in the frontiers of 2-dimensional orbits and points at the ends of the fiber. Our proposed approach to prove \cref{conj:one-and-two-d-orbits} is to first use the preceding corollary to show inaccessible prime ends can only correspond to frontier points at the ends of the fiber and then to directly prove that the frontier points corresponding to frontiers of annulus orbits are locally accessible. Local accessibility of a frontier point is a technical prime ends condition which implies that the frontier is locally connected about that point. The proposed approach to understanding the frontier points corresponding to ends of the fiber is to show that each end has either a unique inaccessible prime end satisfying $X(e)=Y(e)$ or that there is an orbit bundle which strictly contains $\mathcal{B}$.

\subsection{Other Specialized Classes of \texorpdfstring{$\homeo[M]$}{Homeo\_0(M)} Actions}

Similar to the description of faithful $\SO$ actions on closed 3-manifolds in terms of a weighted closed surface, there is a description of faithful $\SO$ actions on closed (oriented) 4-manifolds in terms of a weighted closed 3-manifold \cite{fintushel:ClassificationCircle}\cite{jang:CircleActions}. Similarly, since $\homeo[S_g]$ is algebraically simple for all $g$\cite{bounemoura:SimpliciteGroupes}, any action of $\homeo[T^2]$ on a manifold $M$ restricts to a faithful action of the torus subgroup $\SO\times\SO$. There exists a classical topological classification of such actions on closed 4-manifolds\cite{orlik:ActionsTorusI, orlik:ActionsTorusII}.

\printbibliography

\end{document}